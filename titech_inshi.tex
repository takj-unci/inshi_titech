\documentclass[a4j]{ltjsarticle}
\usepackage{bookmark}
\usepackage{xcolor}
% 数式
\usepackage{luatexja}
\usepackage{amsmath,amsfonts,amssymb,ascmac}
\usepackage{bm}
\usepackage{bbm}
\usepackage{url}
\usepackage{xurl}
\usepackage{mathtools}
\usepackage[shortlabels]{enumitem}
\usepackage{mathrsfs}
\usepackage{tikz}
\usetikzlibrary{cd}
\setcounter{tocdepth}{3}
\newcommand{\Rset}{\mathbb{R}}
\newcommand{\Nset}{\mathbb{N}}
\newcommand{\Zset}{\mathbb{Z}}
\newcommand{\Qset}{\mathbb{Q}}
\newcommand{\Cset}{\mathbb{C}}
\newcommand{\1}{\mathbbm{1}}
\newcommand{\opensub}{\underset{\text{open}}{\subset}}
\newcommand{\closedsub}{\underset{\text{closed}}{\subset}}
\newcommand{\transpose}[1]{\prescript{t\!}{}{#1}}
\DeclareMathOperator*{\interior}{\mathrm{Int}}
\DeclareMathOperator*{\closure}{\mathrm{Cl}}
\DeclareMathOperator*{\supp}{\mathrm{supp}}
\mathtoolsset{showonlyrefs=true}
\numberwithin{equation}{section}
\usepackage{physics}
% 画像
\usepackage{graphicx}

% 定理環境
\usepackage{amsthm}
\theoremstyle{definition}
\newtheorem{thm}{定理}
\newtheorem{cor}[thm]{系}
\newtheorem{dfn}[thm]{定義}
\newtheorem{prop}[thm]{命題}
\newtheorem{lem}[thm]{補題}
\newtheorem{rmk}[thm]{注意}
\newtheorem{eg}[thm]{例}
\newtheorem{prob}[thm]{問題}

\begin{document}

\title{東工大院試}
\author{un cinglé}
\date{\today}
\maketitle
\begin{abstract}
    問題のリンク:
    \begin{quote}
        \url{https://www.math.titech.ac.jp/top/~jimu/Graduate/old-exam/innsi.html}
    \end{quote}
    まだ解けていない問題:
    \begin{itemize}
        \item \hyperref[am_2022_3]{午前,2022[3](3)},
        \item \hyperref[am_2023_3]{午前,2023[3](4)},
    
        \item \hyperref[pm_2021_6]{午後,2021[6](3)},
    \end{itemize}
\end{abstract}
\tableofcontents

\newpage 

\section{午前}
\subsection{2012年度}
問題は
\begin{quote}
    \url{https://www.math.titech.ac.jp/top/~jimu/Graduate/old-exam/H24innsi.pdf}
\end{quote}
を見よ.
\subsubsection*{[1]}
簡単すぎるため省略.
\subsubsection*{[2]}
(1)\ $a>0$を任意に固定する(今は$a=1$でよい).$n\in\Nset_0$に対して$x_n$を,$[an,a(n+1)]$における$f$の最小化元とする.すなわち
\begin{equation}
    an\leq x_n\leq a(n+1)\quad \text{and}\quad f(x_n)=\min_{[an,a(n+1)]}f 
\end{equation}
となる$(x_n)_{n\in\Nset_{0}}$を取る.今
\begin{equation}
    \sum_{n=0}^\infty \int_{an}^{a(n+1)}f(x)\,dx=\int_{0}^\infty f(x)\,dx<\infty 
\end{equation}
なので,とくに
\begin{equation}
    f(x_n)\leq \frac{1}{a}\int_{an}^{a(n+1)}f(x)\,dx\to 0\quad (\text{as $n\to\infty $})
\end{equation}
であるから$f(x_n)\to0\ (n\to\infty)$となる.$x_n\to\infty\ (n\to\infty)$であることはよい.

(2)\ $f(x)=x/(1+x^6\sin^2 x)$は$(\ast)$を満たすが有界ですらない.

(3)\ $\varepsilon>0$を任意に取る.$f$は一様連続であるから,$\delta>0$があり,任意の$x,y\in[0,\infty)$に対して次が成り立つ:
\begin{equation}
    \abs{x-y}\leq \delta\Longrightarrow \abs{f(x)-f(y)}<\varepsilon/2
\end{equation}
ここで,(1)の解答において$a=\delta$としたときの$\qty{x_n}_{n\in\Nset_0}$を取る.$f(x_n)\to 0\ (n\to\infty)$であるから,十分大きな$N\in\Nset$があり,任意の$n\geq N$に対して$f(x_n)<\varepsilon/2$となる.したがって,任意の$x\geq N\delta$に対して$n=\lfloor x/\delta\rfloor $を$x/\delta$の整数部分とすると
\begin{equation}
    f(x)\leq f(x_n)+\abs{f(x)-f(x_n)}<\varepsilon/2+\varepsilon/2=\varepsilon
\end{equation}
を得る($x\in [\delta n,\delta(n+1)]$より$\abs{x-x_n}\leq \delta$に注意せよ).
\subsubsection*{[3]}
(1)\ $\emptyset,\ X\in\mathcal{O}$であることはよい.

$A,B\in \mathcal{O}$だとする.もし$A\subset \Nset$または$B\subset \Nset$ならば$A\cap B\subset \Nset$より$A\cap B\in\mathcal{O}$である.一方,もし$X-A$と$X-B$が$\Nset$の有限部分集合ならば,$X-(A\cap B)=(X-A)\cup(X-B)$もそうである.よって$A\cap B\in \mathcal{O}$.以上よりいずれの場合も$A\cap B\in\mathcal{O}$は成り立つ.

$\qty{A_\lambda}_{\lambda\in\Lambda}$を$\mathcal{O}$の元の族とする.もし任意の$\lambda\in\Lambda$に対して$A_\lambda\subset \Nset$ならば,$\bigcup_{\lambda\in\Lambda}A_\lambda\subset \Nset$より$\bigcup_{\lambda\in\Lambda}A_\lambda\in \mathcal{O}$である.一方,ある$\lambda_0\in\Lambda$について$X-A_{\lambda_0}$が$\Nset$の有限部分集合ならば,$X-\bigcup_{\lambda\in\Lambda}A_\lambda\subset X-A_{\lambda_0}$もそうである.よってこの場合も$\bigcup_{\lambda\in\Lambda}A_\lambda\in\mathcal{O}$が示された.

(2)\ $x,y\in X$を相異なる元とする.このとき最初から$x<y$と仮定してもよい.もし$0=x<y$ならば$A=X-\qty{y},\ B=\qty{y}$が$x$と$y$を分離する開集合たちである.もし$0<x<y$ならば$A=\qty{x},\ B=\qty{y}$でよい.

次に$C,D\subset X$は閉集合で$C\cap D=\emptyset$だとする.このとき,$C\subset \qty{0}$であるか,$C$は$\Nset$の有限部分集合である.$D$についても同様である.もし$C\subset \qty{0}$ならば,$D\subset \Nset$かつ$D$は有限集合だとしてよい.このときは$A=X-D,\ B=D$が$C,D$を分離する.一方,もし$C,D\subset \Nset$らが有限部分集合ならば,$A=C,\ B=D$とおけばよい.

(3)\ 答えははい.$\qty{A_\lambda}_{\lambda\in\Lambda}$を$X$の開被覆だとせよ.このとき$\lambda_0\in\Lambda$で$0\in A_{\lambda_0}$となるものが存在する.ここで$X-A_{\lambda_0}$は$\Nset$の有限部分集合であることに注意する.各$x\in X-A_{\lambda_0}$に対して$\lambda_x\in\Lambda$があり$x\in A_{\lambda_x}$となるので,これら$\qty{A_{\lambda_x}}_{x\in \qty{0}\cup (X-A_{\lambda_0})}$が有限部分開被覆を与える.

(4)\ 丁寧な誘導をありがとう.$f\colon X\to Y$を$f(0)=0,\ f(n)=1/n\ (n\in\Nset)$により定める.$f$は連続であることを示すが,それには$\Rset$の開区間$I=(a,b)$について$f^{-1}(I\cap Y)\in\mathcal{O}$であることを示せばよい.もし$0<a<b$ならば$f^{-1}(I\cap Y)=\qty{n\in\Nset \mid 1/b<n<1/a}\in\mathcal{O}$である.一方,$a\leq 0<b$ならば$f^{-1}(I\cap Y)=\qty{0}\cup \qty{n\in\Nset\mid n>1/b}\in\mathcal{O}$となる.

$f$はコンパクト空間からHausdorff空間への連続全単射であるから同相である.
\subsection{2013年度}
問題は
\begin{quote}
    \url{https://www.math.titech.ac.jp/top/~jimu/Graduate/old-exam/H25innsi.pdf}
\end{quote}
を見よ.
\subsubsection*{[1]}答えは$\pi^2/8a$.極座標に変換する.つまり
\begin{equation}
    x=r\sin\theta\cos\varphi,\quad y=r\sin\theta\cos\varphi,\quad z=r\cos\theta\ ;\quad (r,\theta,\varphi)\in [0,\infty)\times [0,\pi/2]\times [0,\pi/2]
\end{equation}
とおく.すると
\begin{align}
    \iiint_{D}\frac{dxdydz}{(a^2+x^2+y^2+z^2)^2}&=\int_0^\infty\,dr \int_0^{\pi/2}\,d\theta\int_0^{\pi/2}\,d\varphi\,\frac{r^2\sin\theta}{(a^2+r^2)^2}\\
    &=\frac{\pi}{2}\int_{0}^\infty\frac{r^2}{(a^2+r^2)^2}\,dr\\
    &=\frac{\pi}{2}\int_{0}^{\pi/2}\frac{a^2\tan^2 x}{a^4(1+\tan^2x)^2}\frac{a}{\cos^2 x}\,dx\qquad(\text{put $r=a\tan x$})\\
    &=\frac{\pi}{2a}\int_{0}^{\pi/2}\sin^2 x\,dx\\
    &=\frac{\pi}{2a}\qty[\frac{x}{2}-\frac{\sin 2x}{4}]_0^{\pi/2}\\
    &=\frac{\pi^2}{8a}
\end{align}
と計算される.
\subsubsection*{[2]}
(1) 
\begin{equation}
    \min_{I}f\leq \mu_1\coloneq \frac{1}{2a}\int_{-a}^a f(x)\,dx \leq \max_{I}f
\end{equation}
であるから,中間値の定理より$f(b)=\mu_1$となる$b\in I$が存在する.

(2) 
\begin{equation}
    \min_I f=\qty(\frac{3}{2a^3}\min_I f)\int_{-a}^ax^2\,dx\leq \mu_2\coloneq \frac{3}{2a^3}\int_{-a}^a x^2f(x)\,dx\leq \qty(\frac{3}{2a^3}\max_I f)\int_{-a}^ax^2\,dx=\max_I f
\end{equation}
であるから,以下同文.

(3) 部分積分より
\begin{align}
    \mu_3&\coloneq \frac{3}{2a^3}\int_{-a}^a xf(x)\,dx\\
    &=\frac{3(f(a)-f(-a))}{4a}-\frac{3}{4a^3}\int_{-a}^ax^2f'(x)\,dx\\
    &=\frac{3}{4a}\int_{-a}^af'(x)\,dx -\frac{3}{4a^3}\int_{-a}^ax^2f'(x)\,dx\\
    &=\frac{3}{4a^3}\int_{-a}^a (a^2-x^2)f'(x)\,dx 
\end{align}
であり,したがって
\begin{equation}
    \min_{I}f'=\qty(\min_I f')\frac{3}{4a^3}\int_{-a}^a(a^2-x^2)\,dx\leq \mu_3\leq \qty(\max_I f')\frac{3}{4a^3}\int_{-a}^a(a^2-x^2)\,dx=\max_I f'
\end{equation}
であるから,以下同文.
\subsubsection*{[3]}
(1)\ $(\varphi(\bm{0}),\varphi(\bm{0}))=(\bm{0},\bm{0})=0$より$\varphi(\bm{0})=\bm{0}$を得る.

(2)\ $\bm{x},\bm{y}\in \Rset^n,\ a,b\in\Rset $とする.このとき$\varphi(a\bm{x}+b\bm{y})=a\varphi(\bm{x})+b\varphi(\bm{y})$を示せばよいが,内積の双線形性を用いて分解し,さらにわちゃわちゃすると
\begin{align}
    &(\varphi(a\bm{x}+b\bm{y})-a\varphi(\bm{x})-b\varphi(\bm{y}),\varphi(a\bm{x}+b\bm{y})-a\varphi(\bm{x})-b\varphi(\bm{y}))\\
    &=(\varphi(a\bm{x}+b\bm{y}),\varphi(a\bm{x}+b\bm{y}))-a(\varphi(a\bm{x}+b\bm{y}),\varphi(\bm{x}))-b(\varphi(a\bm{x}+b\bm{y}),\varphi(\bm{y}))\\
    &\quad -a(\varphi(\bm{x}),\varphi(a\bm{x}+b\bm{y}))+a^2(\varphi(\bm{x}),\varphi(\bm{x}))+ab(\varphi(\bm{x}),\varphi(\bm{y}))\\
    &\quad -b(\varphi(\bm{y}),\varphi(a\bm{x}+b\bm{y}))+ab(\varphi(\bm{y}),\varphi(\bm{x}))+b^2(\varphi(\bm{y}),\varphi(\bm{y}))\\
    &=(a\bm{x}+b\bm{y},a\bm{x}+b\bm{y})-a(a\bm{x}+b\bm{y},\bm{x})-b(a\bm{x}+b\bm{y},\bm{y})\\
    &\quad -a(\bm{x},a\bm{x}+b\bm{y})+a^2(\bm{x},\bm{x})+ab(\bm{x},\bm{y})\\
    &\quad -b(\bm{y},a\bm{x}+b\bm{y})+ab(\bm{y},\bm{x})+b^2(\bm{y},\bm{y})\\
    &=0
\end{align}
となることから従う.

(3)\ $\bm{e}_1,\ldots,\bm{e}_n$を$\Rset^n$の標準基底とする.$\varphi\colon \Rset^n\to \Rset^n$は線形写像だから行列$A=\begin{bmatrix}
    \varphi(\bm{e}_1) & \cdots & \varphi(\bm{e}_n)
\end{bmatrix}$により$\varphi(\bm{x})=A\bm{x}$と表される.仮定より$\varphi(\bm{e}_1),\ldots,\varphi(\bm{e}_n)$は$\Rset^n$の正規直交系を与えており,したがって$A$は直交行列である.

\subsubsection*{[4]}
(1)\ 略.

(2)\ 行列$\begin{bmatrix}
    \mathbf{a} & \mathbf{b} & \mathbf{c} & \mathbf{d}
\end{bmatrix}$を基本変形してなんやかんややると$X+Y;\ X\cap Y$の基底としてそれぞれ$\mathbf{a},\mathbf{b},\mathbf{c};\ \mathbf{a}-\mathbf{b}$が取れることがわかる.

\subsubsection*{[5]}
(1)\ $(X,d)$は,$n=1$のとき不連結,$n\geq2$のとき連結.

(2)\ $(X,d)$の有界閉集合$B=\qty{\bm{p}\in X\mid \abs{\bm p}\leq 1}$はコンパクトではない.実際,
\begin{equation}
    B_n=\qty{\bm{p}\in X\mid  1/n< \abs{\bm{p}}}
\end{equation}
で定まる$A$の開被覆$\qty{B_n}_{n\in\Nset}$は,有限部分被覆を持たない.

(3)\ はい.$X$は完備距離空間$\Rset \times S^{n-1}$に同相である.ここで$S^{n-1}$は$\Rset^n$内の単位$(n-1)$-球面.実際に
\begin{equation}
    X\ni \bm{p}\mapsto(\log \abs{p},\ p/\abs{p})\in \Rset\times S^{n-1}
\end{equation}
が同相写像である.
\paragraph*{余談}ある完備距離空間と同相となるような位相空間をポーランド空間という.例えば,ポーランド空間の$G_\delta$集合はポーランド空間であることが知られている.とくに無理数全体の集合に$\Rset$の相対位相を入れるとポーランド空間になる.
\subsection{2014年度}
問題は
\begin{quote}
    \url{https://www.math.titech.ac.jp/top/~jimu/Graduate/old-exam/H26innsi.pdf}
\end{quote}
を見よ.
\subsubsection*{[1]}
答えは$\pi/12$.円柱座標
\begin{equation}
    x=r\cos\theta,\quad y=r\sin\theta,\quad \zeta=z\ ;\quad (r,\theta,\zeta)\in [0,\infty)\times [0,\pi/2]\times [0,\infty)
\end{equation}
を用いよ.
\subsubsection*{[2]}
(1)\ 答えは$p\leq 3$で,$p<3$のとき$f(x)=0$;$p=3$のとき$f(x)=\begin{cases}
    0 & (\text{if $x=0$})\\
    x^{-2} & (\text{if $0<x\leq 1$})
\end{cases}$である.

(2)\ 答えは$p<1$.$p=3$のときは明らかに$f_n$は一様収束しない(連続関数の一様収束極限が連続であることを思い出せばよい).そこで$p<3$だとしよう.
\begin{equation}
    f'_n(x)=\frac{n^p(1-2n^3x^3)}{(1+n^3x^3)^2}
\end{equation}
であり,増減を考えると$f_n(x)$は$x=2^{-1/3}n^{-1}$で最大値をとることがわかる.よって
\begin{equation}
    \sup_{[0,1]}\abs{f_n}=f_n(2^{-1/3}n^{-1})=\frac{2^{2/3}\cdot n^{p-1}}{3}
\end{equation}
である.よって$f_n$が一様に$f=0$に収束するのは$p<1$のときに限る.

(3)\ 答えは$p\leq 2$.変数変換$y=nx$により
\begin{equation}
    \int_0^1f_n(x)\,dx=n^{p-2}\int_0^n\frac{y}{1+y^3}\,dy
\end{equation}
だが,
\begin{equation}
    I_n\coloneq \int_0^n\frac{y}{1+y^3}\,dy\leq \int_0^1y\,dy+\int_1^n\frac{1}{y^2}\,dy\leq \frac{3}{2}-\frac{1}{n}\leq \frac{3}{2}<\infty 
\end{equation}
は収束するので$n^{p-2} I_n$は$p\leq 2$のときに限って収束する.
\subsubsection*{[3]}
(1)\ 例えば
\begin{equation}
    E_1=\begin{bmatrix}
        1 & 0\\
        0 & 0
    \end{bmatrix},\quad E_2=\begin{bmatrix}
        0 & 1\\
        0 & 0
    \end{bmatrix},\quad E_3=\begin{bmatrix}
        0 & 0\\
        1 & 0
    \end{bmatrix},\quad E_4=\begin{bmatrix}
        0 & 0\\
        0 & 1
    \end{bmatrix}
\end{equation}

(2)\ $(a+d)^2$.

(3)\ $16$.

\subsubsection*{[4]}
(1)\ $F$はスケール不変である.つまり$F(k\bm{x})=F(\bm{x})\ (\forall k\in\Rset\setminus\qty{0},\ \bm{x}\in \Rset^3\setminus\qty{\bm{0}})$.とくに$F(\bm{x})=F(\bm{x}/\abs{\bm{x}})$.

(2)\ $F$の連続性より,$F$の$S^2$への制限は最大値・最小値を持つが,(1)よりこれらは$\Rset^3\setminus\qty{\bm{0}}$全域における$F$の最大値・最小値を与える.

(3)\ 仮定より$\bm{a}$と$\bm{b}$は一次独立.よって$\det P=\det\begin{bmatrix}
    \bm{a} & \bm{b} & \bm{a}\times \bm{b}
\end{bmatrix}=\abs{\bm{a}\times \bm{b}}^2\neq0$なので$P$は正則である.さて,$\bm{a},\bm{b}$が$A$の固有ベクトルであることを示そう.$\bm{a},\bm{b}$はそれぞれ,拘束条件$\abs{\bm{x}}^2=1$のもとでの
\begin{equation}
    \langle A\bm{x},\bm{x}\rangle =\sum_{i,j}a_{ij}x_ix_j 
\end{equation}
の最大化・最小化元である.ここで$A=(a_{ij})_{i,j},\ \bm{x}=\transpose{(x_1,x_2,x_3)}$とおいた.Lagrangeの未定乗数法より,$\bm{x}=\bm{a},\bm{b}$に対して
\begin{equation}
    \sum_{j}a_{ij}x_j-2\lambda x_i=0\quad (\text{for $i=1,2,3$})
\end{equation}
が成り立つ.ここで$\lambda$は未定定数を表す.これは$\bm{a},\bm{b}$が$A$の固有ベクトルであることを示している.最後に$\bm{c}$が$A$の固有ベクトルであることを示す.$A$は対称行列であり,$\bm{a}$は$A$の固有ベクトルなので固有値を$\lambda$として
\begin{equation}
    \langle A\bm{c},\bm{a}\rangle = \langle \bm{c},A\bm{a}\rangle = \lambda \langle \bm{c},\bm{a}\rangle =0
\end{equation}
が成り立つ.同様に$\langle A\bm{c},\bm{b}\rangle=0$が成り立つので,$A\bm{c}$は$\bm{a},\bm{b}$に直交するベクトルである.よって$A\bm{c}$は$\bm{a}\times \bm{b}=\bm{c}$の定数倍になるしかない.

以上より$\bm{a},\bm{b},\bm{c}$が$A$の固有ベクトルからなる$\Rset^3$の基底をなすことが示された.

\subsubsection*{[5]}
(1)\ はい.$a,b\in\Rset,\ a\neq b$とする.最初から$a<b$としてよい.このとき$[a,b)$と$[b,b+1)$は$a$と$b$を分離する開集合である.

(2)\ いいえ.$U\coloneq(-\infty,0)=\bigcup_{n\geq0}[-n,0)\in\mathcal{O}$および$V\coloneq [0,\infty)=\bigcup_{n\geq0}[0,n)\in\mathcal{O}$は空でなく,$U\cup V=\Rset,\ U\cap V=\emptyset$.

(3)\ いいえ.$[0,1]=\bigcup_{n\in\Nset}[0,1-1/2n)\cup [1,2)$は有限部分被覆を持たない.
\subsection{2015年度}
問題は
\begin{quote}
    \url{https://www.math.titech.ac.jp/top/~jimu/Graduate/old-exam/H27innsi.pdf}
\end{quote}
を見よ.
\subsubsection*{[1]}
(1)\ 答えは$\alpha<-1$.極座標に変換すると
\begin{equation}
    2\pi\int_0^\infty (1+r^2)^\alpha r\,dr 
\end{equation}
の収束・発散を見ればよい.

(2)\ $M=\sup_{n}\abs{a_n}<\infty$とおく.$\varepsilon>0$とせよ.$\lim_{n}a_n=a$より,十分大きい$N\in\Nset$があり,任意の$n\geq N$に対して$\abs{a-a_n}\leq \varepsilon$が成り立つ.よって$n\geq N$ならば
\begin{equation}
    \abs{a-\frac{a_1+\cdots+a_n}{n}}\leq \frac{\abs{a-a_1}+\cdots+\abs{a-a_n}}{n}\leq \frac{N(\abs{a}+M)}{n}+\frac{(n-N)\varepsilon}{n}
\end{equation}
よって
\begin{equation}
    \limsup_{n}\abs{a-\frac{a_1+\cdots+a_n}{n}}\leq \limsup_{n}\qty(\frac{N(\abs{a}+M)}{n}+\frac{(n-N)\varepsilon}{n})=\varepsilon
\end{equation}
となる.$\varepsilon>0$は任意なので$\lim_{n}\frac{a_1+\cdots+a_n}{n}=a$を得る.
\subsubsection*{[2]}
(1)\ $1>0$に対して$M>0$があり,任意の$\abs{x}\geq M$に対して$\abs{f(x)}\leq 1$となる.よって
\begin{equation}
    \sup_{\Rset}\abs{f}\leq 1+\sup_{[-M,M]}\abs{f}<\infty 
\end{equation}
である.

(2)\ $\varepsilon>0$とせよ.仮定より,$M>0$があり,任意の$\abs{x}\geq M$に対して$\abs{f(x)}\leq \varepsilon/3$となる.$f$は有界閉区間$[-M,M]$上一様連続であるから,$0<\delta<1$があり,任意の$x,y\in [-M,M]$に対して
\begin{equation}
    \abs{x-y}\leq \delta\Longrightarrow \abs{f(x)-f(y)}\leq \varepsilon/3 
\end{equation}
が成り立つ.$x,y\in\Rset$が$\abs{x-y}\leq \delta$を満たすと仮定しよう.
\begin{itemize}
    \item $x,y\in[-M,M]$ならば$\abs{f(x)-f(y)}\leq \varepsilon/2$である.
    \item $x,y\notin [-M,M]$ならば$\abs{f(x)-f(y)}\leq \abs{f(x)}+\abs{f(y)}\leq \varepsilon$である.
    \item $x>M$かつ$y\in [-M,M]$ならば,$\abs{y-M}\leq \abs{x-y}\leq \delta$より$\abs{f(x)-f(y)}\leq \abs{f(x)}+\abs{f(M)}+\abs{f(M)-f(y)}\leq \varepsilon$である.同様に$\abs{x}\geq M$もしくは$\abs{y}\geq M$の場合が示せる.
\end{itemize}

(3)\ $M=\sup_{\Rset}\abs{f}$とおく.$z=ny$と変数変換して
\begin{equation}
    f_n(x)=\int_{-\infty}^{\infty} f(x+z/n)g(z)\,dz 
\end{equation}
となる.
\begin{align}
    \abs{f(x)-f_n(x)}\leq \int_{-\infty}^\infty \abs{f(x)-f(x+z/n)}g(z)\,dz 
\end{align}
である.さて,$\varepsilon>0$とせよ.$f$は一様連続なので,$\delta>0$があり,任意の$x,y\in \Rset$に対して
\begin{equation}
    \abs{x-y}\leq \delta\Longrightarrow \abs{f(x)-f(y)}\leq \varepsilon 
\end{equation}
となる.よって$x\in\Rset$に対して
\begin{align}
    \abs{f(x)-f_n(x)}&\leq \int_{\abs{z}\leq n\delta}\abs{f(x)-f(x+z/n)}g(z)\,dz+\int_{\abs{z}\geq n\delta} \abs{f(x)-f(x+z/n)}g(z)\,dz\\
    &\leq \varepsilon\int_{\Rset}g(z)\,dz+2M\int_{\abs{z}\geq n\delta}g(z)\,dz \\
    &\leq \varepsilon+\int_{\abs{z}\geq n\delta}g(z)\,dz 
\end{align}
だが,今$\int_{\abs{z}\geq n\delta}g\,dz\to \infty \ (n\to\infty)$より 
\begin{equation}
    \limsup_{n}\sup_{x\in\Rset}\abs{f(x)-f_n(x)}\leq \varepsilon 
\end{equation}
となる.$\varepsilon>0$は任意なので$\sup_{x\in\Rset}\abs{f(x)-f_n(x)}\to0\ (n\to\infty)$を得る.
\subsubsection*{[3]}
固有多項式は$(\lambda-a-b)^2(\lambda-a+b)(\lambda+a-b)$.固有値$\lambda$に対する固有空間を$V_\lambda$と書くことにすると,
\begin{equation}
    V_{2a}=\left\langle\begin{bmatrix}
        1 \\
        0 \\
        0 \\
        1 
    \end{bmatrix},\ \begin{bmatrix}
        0 \\
        1 \\
        1 \\
        0
    \end{bmatrix}\right\rangle,\quad V_{0}=\left\langle\begin{bmatrix}
        1 \\
        0 \\
        0 \\
        -1 
    \end{bmatrix},\ \begin{bmatrix}
        0 \\
        1 \\
        -1 \\
        0
    \end{bmatrix}\right\rangle\quad (\text{if $a=b$})
\end{equation}
\begin{equation}
    V_{a+b}=\left\langle\begin{bmatrix}
        1 \\
        0 \\
        0 \\
        1 
    \end{bmatrix},\ \begin{bmatrix}
        0 \\
        1 \\
        1 \\
        0
    \end{bmatrix}\right\rangle,\ V_{a-b}=\left\langle\begin{bmatrix}
        1 \\
        0 \\
        0 \\
        -1 
    \end{bmatrix}\right\rangle,\ V_{-a+b}\left\langle\begin{bmatrix}
        0 \\
        1 \\
        -1 \\
        0
    \end{bmatrix}\right\rangle\quad (\text{if $a\neq b$})
\end{equation}

最小多項式は
\begin{equation}
    (\lambda-2a)\lambda\quad (\text{if $a=b$})\quad ; \quad (\lambda-a-b)(\lambda-a+b)(\lambda+a-b) \quad (\text{if $a\neq b$})
\end{equation}
\subsubsection*{[4]}
(1)\ $\ker(E_n-A)\subset \mathop{\text{Im}}(A)$より従う.(2)\ 等号成立は$\ker(E_n-A)\supset \mathop{\text{Im}}(A)$つまり$A-A^2=(E_n-A)A=0$と同値.
\subsubsection*{[5]}
(1)\ $\emptyset,\Rset^2\in\mathcal{O}$はよい.$U_\lambda,U,V\subset \Rset^2$を$\Rset$の開集合とせよ($\lambda\in\Lambda$).このとき
\begin{equation}
    (U\times \Rset)\cap (V\times \Rset)=(U\cap V)\times \Rset,\quad \bigcup_{\lambda\in\Lambda}(U_\lambda\times \Rset)=\qty(\bigcup_{\lambda\in\Lambda}U_{\lambda})\times \Rset\in\mathcal{O}
\end{equation}
である.

(2)\ いいえ.$(0,0),\ (0,1)\in\Rset^2$は開集合で分離できない.

(3)\ コンパクトであるものは,$J\times J,\ I\times I$のみ.ほかはコンパクトでない.

(4)\ $\mathop{\text{cl}}(I\times I)=J\times \Rset,\ \mathop{\text{cl}}(I\times J)=J\times \Rset,\ \mathop{\text{cl}}(J\times I)=J\times \Rset,\ \mathop{\text{cl}}(J\times J)=J\times \Rset$
\subsection{2016年度}
問題は
\begin{quote}
    \url{https://www.math.titech.ac.jp/top/~jimu/Graduate/old-exam/H28innsi.pdf}
\end{quote}
を見よ.
\subsubsection*{[1]}
(1),(2)\ 略.

(3)\ $X=(x_{ij})\in M_n(\Cset)$に対して
\begin{equation}
    D_A(X)=O\iff x_{11}=x_{33},\ x_{12}=x_{32},\ x_{13}=x_{31},\ x_{21}=x_{23}
\end{equation}
であることがわかる.よって$\dim \ker(D_A)=5$,次元定理より$\dim \Im(D_A)=4$.

\subsubsection*{[2]}
(1)\ $\Phi_n(\lambda)=A-\lambda$を$A$の固有多項式とすると,
\begin{equation}
    \Phi_1(\lambda)=1-\lambda,\ \Phi_2(\lambda)=\lambda^2-2\lambda\quad ;\quad \Phi_n(\lambda)=(1-\lambda)\Phi_{n-1}(\lambda)-\Phi_{n-2}(\lambda)\quad (n\geq3).
\end{equation}
を得る.よって
\begin{equation}
    \det A=\Phi_n(0)=\begin{cases}
        1 & (\text{if }n\equiv 0,1\mod 6),\\
        -1 & (\text{if }n\equiv 3,4\mod 6),\\
        0  & (\text{if }n\equiv 2,5\mod 6).
    \end{cases}
\end{equation}

(2)\ direct calculationより固有空間は
\begin{equation}
    \left\langle\transpose{(1,0,-1,0,\ldots,0,(-1)^{(n-1)/2})}\right\rangle.
\end{equation}
\subsubsection*{[3]}
(1) 略.

(2) $(-\infty,0)$はコンパクトでないが$(-\infty,0]$はコンパクトである.$(-\infty,0)$の開被覆$\qty{(-\infty,-1/n)\mid n\in\Nset}$は有限部分被覆を持たない.$\qty{U_\lambda}_{\lambda\in\Lambda}$を$(-\infty,0]$の$\mathcal{O}$-開被覆とせよ.$U_\lambda=(-\infty,a_\lambda)$とおく($U_\lambda=\emptyset$ならば$a_\lambda=-\infty$,$U_\lambda=\Rset$ならば$a_{\lambda}=\infty$と解釈せよ).$0\in \bigcup_{\lambda\in\Lambda} U_{\lambda}$であるから,ある$\lambda_{0}\in\Lambda$に対して$0\in U_{\lambda_0}$である.$\qty{U_{\lambda_0}}$が有限部分被覆.

(3) 答えははい.$(0,1)\cup(2,3)\subset U\cup V;\ U,V\in\mathcal{O};\ U\cap V\cap((0,1)\cup(2,3))=\emptyset$とせよ.このとき$U\subset V$または$U\supset V$が成り立つので,$U\cap((0,1)\cup(2,3))=\emptyset$または$V\cap ((0,1)\cup(2,3))=\emptyset$である.よって$(0,1)\cup(2,3)$は連結.

(4) $f\colon(\Rset,\mathcal{O})\to(\Rset,\mathcal{O})$を連続写像とする.$x,x'\in\Rset$で$x\leq x',\ f(x)>f(x')$を満たすものがあったと仮定しよう.$f$の連続性より
\begin{equation}
    f^{-1}((-\infty,f(x)))=(-\infty,a),\quad a\in\Rset\cup\qty{\pm\infty}
\end{equation}
となる$a$がある.$f(x')<f(x)$より$x'<a$である.一方明らかに$x\geq a$である.よって$x'<x$となり矛盾.

\subsubsection*{[4]}
(1) 答えは$p+q>0$.$x\neq0$に対して$f_{p,q}(x)=\abs{x}^{p+q}\abs{\sin x/x}^q$であり,$\sin x/x\to 1\ (x\to0)$に注意するとわかる.

(2) 答えは$p+q>1$.$x\neq0$に対して
\begin{equation}
    \frac{f_{p,q}(x)-f_{p,q}(0)}{x}=\frac{x}{\abs{x}}\abs{x}^{p+q-1}\abs{\frac{\sin x}{x}}^q
\end{equation}
が$x\to0$で有限確定値に収束する条件を考えればわかる.

(3) 答えは$p+q>-1$.極座標に変換すると
\begin{equation}
    \iint_{D}f_{p,q}(\sqrt{x^2+y^2})\,dxdy =\frac{\pi}{2}\int_{0}^1r^{p+q+1}\abs{\frac{\sin r}{r}}^q\,dr
\end{equation}
で同様.$[0,1]$で$0<\exists m\leq \abs{\sin r/r}\leq 1$に注意せよ.

\subsubsection*{[5]}
(1) まず$f,g$は連続関数の一様収束極限として連続である.$\delta>0$を固定する.ある$N$以上の任意の$n$について$\abs{a-a_n}<\delta$となる.よって$n\geq N$に対して
\begin{equation}
    \abs{f(a)-f_n(a_n)}\leq \abs{f(a)-f(a_n)}+\abs{f(a_n)-f_n(a_n)}\leq \sup_{b\in \Rset,\ \abs{a-b}<\delta}\abs{f(a)-f(b)}+\sup_{b\in\Rset}\abs{f(b)-f_n(b)}
\end{equation}
$\limsup_{n\to\infty}$を取って($f_n$の一様収束性より)
\begin{equation}
    \limsup_{n\to\infty} \abs{f(a)-f(a_n)}\leq \sup_{b\in\Rset,\ \abs{a-b}<\delta}\abs{f(a)-f(b)}+0
\end{equation}
$\limsup_{\delta\to+0}$を取って($f$の点$a$での連続性より)
\begin{equation}
    \limsup_{n\to\infty}\abs{f(a)-f_n(a_n)}\leq \limsup_{\delta\to+0}\sup_{b\in\Rset,\ \abs{a-b}<\delta}\abs{f(a)-f(b)}=0
\end{equation}
を得る.

(2) $g_n$は一様収束列なので一様有界である.よって,$M\coloneq\sup_{n\geq1}\sup_{x\in[0,1]}\abs{g_n(x)}+\sup_{x\in[0,1]}\abs{g(x)}<\infty$となる.$f$は有界閉区間$[-M,M]$上一様連続である.$\delta>0$を固定する.$N\in\Nset$があり,$\sup_{n\geq N}\sup_{x\in[0,1]}\abs{g(x)-g_n(x)}<\delta$となる.よって
\begin{equation}
    \sup_{n\geq N}\sup_{x\in[0,1]}\abs{f(g(x))-f(g_n(x))}\leq \sup_{a,b\in[-M,M],\ \abs{a-b}<\delta}\abs{f(a)-f(b)}
\end{equation}
よって
\begin{equation}
    \limsup_{n\to\infty}\sup_{x\in[0,1]}\abs{f(g(x))-f(g_n(x))}\leq \sup_{a,b\in[-M,M],\ \abs{a-b}<\delta}\abs{f(a)-f(b)}
\end{equation}
となり,$\delta\to0$とすることで結論を得る($f$の$[-M,M]$における一様連続性より).

(3) 
\begin{align}
    &\quad \limsup_{n\to\infty}\sup_{x\in[0,1]}\abs{f(g(x))-f_n(g_n(x))}\\
    &\leq \limsup_{n\to\infty}\sup_{x\in[0,1]}\abs{f(g(x))-f(g_n(x))}+\limsup_{n\to\infty}\sup_{x\in[0,1]}\abs{f(g_n(x))-f_n(g_n(x))}\\
    &\leq 0+\limsup_{n\to\infty}\sup_{a\in\Rset}\abs{f(a)-f_n(a)}\\
    &\leq 0+0
\end{align}
\subsection{2017年度}
問題は
\begin{quote}
    \url{https://www.math.titech.ac.jp/top/~jimu/Graduate/old-exam/H29innsi.pdf}
\end{quote}
を見よ.
\subsubsection*{[1]}
(1) 略.

(2) 結果だけ書く:
\begin{equation}
    P^{-1}AP=\begin{bmatrix}
        1 & & \\
         & 2 & \\
         & & 3
    \end{bmatrix},\quad \text{where}\quad P=\begin{bmatrix}
        1 & 0 & 1 \\
        0 & 1 & 0 \\
        -1 & 0 & 1 
    \end{bmatrix}
\end{equation}

(3) $C_{V}(A)=\qty{X\in V\mid XA=AX}=W$と書く.$C_{V}(P^{-1}AP)=P^{-1}WP\cong W$だが,$P^{-1}AP$は相異なる固有値を持つ対角行列であるから,$C_{V}(P^{-1}AP)$は対角行列全体.したがって$\dim W=3$.線形写像$\Rset[t]\to W,\quad f(t)\mapsto f(X)$を考察する.kernelは$A$の最小多項式$(t-1)(t-2)(t-3)$で生成されるイデアルである.準同型定理より
\begin{equation}
    \Rset^3 \cong \Rset[t]/((t-1)(t-2)(t-3))\cong \Rset[A]\subset W 
\end{equation}
である.$\dim W=3$だったので$W=\Rset[A]$を得る.
\subsubsection*{[2]}
$\Im(A^{n})\subset \Im(A^{n+1})$が示されれば,逆の包含は明らかなので結論が従う.部分空間の列
\begin{equation}
    \Im(A^{n+1})\subset \Im(A^{n})\subset \cdots\subset \Im(A^{2})\subset \Im(A)\subset \Im(A^0)=\Cset^n
\end{equation}
を考えると,$\dim \Cset^n=n$より,ある$k\in\qty{0,1,\ldots,n}$で$\Im(A^{k+1})=\Im(A^k)$となる.よって
\begin{equation}
    \Im(A^{n+1})=\Im(A^{n-k}\cdot A^{k+1})=\Im(A^{n-k}\cdot A^{k})=\Im(A^n)
\end{equation}
を得る.
\subsubsection*{[3]}
(1) 
\begin{equation}
    \mathcal{O}=\qty{U\subset \Rset\mid \forall x\in U\ \exists U(a,r)\in \mathcal{B}\ \text{s.t. }x\in U(a,r)\subset U}
\end{equation} 
が$\Rset$の位相であることが示されれば,$\mathcal{B}$がその開基であることは明らかである.省略.

(2) 「$x\in U\in \mathcal{O}$ならば$-x\in U$」であることに注意する.すると$1,-1\in\Rset$は$\mathcal{O}$の開集合で分離できない.

(3) $\mathcal{O}$は通常のEuclid位相$\mathcal{O}_{\Rset}$よりも弱い位相であることから従う.

(4) $[0,1]$は$\mathcal{O}_{\Rset}$の位相でコンパクトであるから,$\mathcal{O}$でもコンパクトである.$-1\in \Rset\setminus[0,1]$だが$1\notin \Rset\setminus[0,1]$である.よって$\Rset\setminus[0,1]$は開集合ではない.
\subsubsection*{[4]}
非負値単調非増加数列$\qty{a_n}_{n}$について,
\begin{equation}
    \text{$\sum_{n}a_n$が収束する}\iff \text{$\sum_{n}2^n a_{2^n}$が収束する}
\end{equation}
であることに注意する.よって$\sum_{n\geq2}1/(n^p\log n)$が収束することは$\sum_{n}2^n\cdot 1/(2^{np}n)=\sum_{n}n^{-1}2^{(1-p)n}$が収束することと同値であり,これは$\sum_{n}2^n\cdot 2^{-n}2^{(1-p)2^{n}}=\sum_{n}2^{(1-p)2^n}$が収束することと同値でこれは$p>1$と同値.

(2) 仮定より,十分大きな$N\in\Nset$があり,任意の$n,m\geq N$に対して$\sup_{x\in\Rset}\abs{f_m(x)-f_n(x)}\leq 1$となる.各$f_n$は多項式であるから,$n\geq N$に対して,$f_n$の次数は$n$によらない定数であり,$d\geq1$について,$f_n\ (n\geq N)$の$d$次の係数は$n$によらない.$f_n$がある$f$に一様収束するということから定数項はある実数に収束し,したがって$f$は多項式となる.

\subsubsection*{[5]}
(1)\ $\varepsilon>0$とする.仮定より,$\delta_1>0$があり
\begin{equation}
    \abs{f(x)-f(y)}\leq \varepsilon\quad (x,y\in [1-\delta_1,1))
\end{equation}
となる.$f$は有界閉区間$[0,1-\delta_1/2]$で一様連続であるから,ある$\delta_2>0$に対して
\begin{equation}
    \abs{f(x)-f(y)}\leq \varepsilon\quad (x,y\in [0,1-\delta_1/2],\ \abs{x-y}\leq \delta_2)
\end{equation}
となる.さて,$x,y\in[0,1),\ 0\leq y-x\leq \min\qty{\delta_1/2,\delta_2}$とする.もし$x\in [0,1-\delta_1]$ならば$y\in[0,1-\delta_1/2]$であるから$\abs{x-y}\leq \delta_2$とあわせて$\abs{f(x)-f(y)}\leq\varepsilon$である.$x\in [1-\delta_1,1)$ならば$y\in[1-\delta_1,1)$でもあるから$\abs{f(x)-f(y)}\leq\varepsilon$である.

(2) $\limsup_{x\to 1-0}f(x)>t>s>\liminf_{x\to1-0}f(x)$となる$s,t$を取る.すると,任意の$x\in[0,1)$に対して,$y,z\in(x,1)$があり,$f(y)<s<t<f(z)$となる.よって$f(z)-f(y)>t-s>0$.これは$f$が一様連続ではないことを示している.

(3) 仮定より$(1-t)^\alpha f'(t)$は$[0,1]$上の連続関数に一意に拡張され,とくに有界である.$M=\sup_{t\in [0,1)}(1-t)^\alpha \abs{f'(t)}<\infty$とおく.$x,y\in [0,1)$に対して
\begin{equation}
    \abs{f(x)-f(y)}=\abs{\int_{x}^{y} (1-t)^{-\alpha}(1-t)^\alpha f'(t)\,dt}\leq M\abs{\int_{x}^{y}(1-t)^{-\alpha}\,dt}=\frac{M}{1-\alpha}\abs{(1-x)^{1-\alpha}-(1-y)^{1-\alpha}} 
\end{equation}
より一様連続性が従う.
\subsection{2018年度}
問題は
\begin{quote}
    \url{https://www.math.titech.ac.jp/top/~jimu/Graduate/old-exam/H30innsi.pdf}
\end{quote}
を見よ.
\subsubsection*{[1]}
(1) 略.(2) $T(x^3)=3x^3+3x^2y,\ T(x^2y)=x^3+3x^2y+2xy^2,\ T(xy^2)=2xy^2+3xy^2+y^3,\ T(y^3)=3xy^2+3y^3$より$T$の行列表示は
\begin{equation}
    A=\begin{bmatrix}
        3 & 1 & 0 & 0 \\
        3 & 3 & 2 & 0 \\
        0 & 2 & 3 & 3 \\
        0 & 0 & 1 & 3 
    \end{bmatrix}
\end{equation}
である.固有多項式を頑張って計算すると
\begin{equation}
    \det(\lambda - T)=\lambda(\lambda-2)(\lambda-4)(\lambda-6)
\end{equation}
であり,各$\lambda\in\qty{0,2,4,6}$に対応する固有空間$V_{\lambda}$は
\begin{equation}
    V_{0}=\langle(x-y)^3\rangle,\quad V_{2}=\langle (x+y)(x-y)^2\rangle,\quad V_{4}=\langle (x+y)^2(x-y)\rangle,\quad V_{6}=\langle(x+y)^3\rangle
\end{equation}
よって$V=V_{0}\oplus V_{2}\oplus V_{4}\oplus V_{6}$の基底が得られた.
\subsubsection*{[2]}
(1) $\ker f$の補空間$W_{0}$を取る.つまり$V=\ker f\oplus W_{0}$である.$W_{0}$は$r$次元で$f|_{W_{0}}$は単射($W_{0}\cap \ker f=0$である).

(2) $W_{0}$の基底$w_1,\ldots,w_r$を取る.与えられた$x\in\Rset^n$を$x=v+w\ (v\in\ker f,\ w\in W_{0})$と分解せよ.このとき$W'=\langle v+w_1,\ldots,v+w_r\rangle$とおくと$v\in W'$,よって$x\in W_{0}+W'$である.$W'\in S$であることも簡単にわかる.
\subsubsection*{[3]}
(1) 略.(2) $(0,0)$と$(1,0)$が開集合で分離できない.(3) 易しい.(4) $\mathcal{O}$から定まる位相の閉集合系は
\begin{equation}
    \qty{C\subset X\mid (0,0)\in C}\cup \mathcal{P}(\qty{(1,0),(-1,0),(0,1),(0,-1)})
\end{equation}
であることに注意する.$Q\subset X$が可算ならば$Q\cup \qty{(0,0)}$は$Q$を含む閉集合であるから$\overline{Q}\subset Q\cup \qty{(0,0)}$.$X$は非可算集合なので$\overline{Q}=X$とはなり得ない.

\subsubsection*{[4]}
(1) $f$は有界なので$M\coloneq \sup_{[0,\infty)}\abs{f}$とおく.$0<s<t$に対して
\begin{align}
    \abs{\alpha-g(t)}&\leq \abs{\frac{1}{t}\int_{0}^t (\alpha-g(t))\,dx}\\
    &\leq \frac{1}{t}\qty{\int_{0}^s \abs{\alpha-f(x)}\,dx+\int_{s}^t\abs{\alpha-f(x)}\,dx}\\
    &\leq \frac{(\abs{\alpha}+M)s}{t}+\frac{(t-s)}{t}\sup_{x\geq s}\abs{\alpha-f(x)}
\end{align}
$\limsup_{t\to\infty}$を取ると
\begin{equation}
    \limsup_{t\to\infty}\abs{\alpha-g(t)}\leq \sup_{x\geq s}\abs{\alpha-f(x)}
\end{equation}
$s\to\infty$として結論を得る.

(2) 答えは$\alpha/2$.$0<s<t$に対して
\begin{align}
    \abs{\frac{\alpha}{2}-h(t)}&= \frac{1}{t^2}\abs{\int_{0}^tx(\alpha-f(x))\,dx}\\
    &\leq \frac{1}{t}\qty{\int_0^s \abs{\alpha-f(x)}\,dx+\int_s^t \abs{\alpha-f(x)}\,dx} \\
    &\leq \cdots 
\end{align}
あとは同様.
\subsubsection*{[5]}
$(x+y)/2$まわりのTaylor展開を考える.
\begin{align}
    f(x)&=f\qty(\frac{x+y}{2})+\frac{x-y}{2}f\qty(\frac{x+y}{2})+\frac{(x-y)^2}{8}f''\qty(\frac{x+y}{2})+\int^{x}_{\frac{x+y}{2}}dt\int^{t}_{\frac{x+y}{2}}ds\qty(f''(s)-f''\qty(\frac{x+y}{2})),\\
    f(y)&=f\qty(\frac{x+y}{2})+\frac{y-x}{2}f\qty(\frac{x+y}{2})+\frac{(y-x)^2}{8}f''\qty(\frac{x+y}{2})+\int^{y}_{\frac{x+y}{2}}dt\int^{t}_{\frac{x+y}{2}}ds\qty(f''(s)-f''\qty(\frac{x+y}{2}))
\end{align}
である.辺々足して
\begin{equation}
    f(x)-2f\qty(\frac{x+y}{2})+f(y)=\frac{(x-y)^2}{4}f''\qty(\frac{x+y}{2})+\qty(\int^{x}_{\frac{x+y}{2}}dt\int^{t}_{\frac{x+y}{2}}ds+\int^{y}_{\frac{x+y}{2}}dt\int^{t}_{\frac{x+y}{2}}ds)\qty(f''(s)-f''\qty(\frac{x+y}{2}))
\end{equation}
を得る.両辺を$(x-y)^2/4$で割って$x,y\to a$とすれば結論が得られる.具体的には次のようにする:任意の$\varepsilon>0$に対して$\delta>0$があり,$\abs{s-a}\leq \delta$に対して$\abs{f''(x)-f''(s)}< \varepsilon$となる.そこで$a-\delta<x<a<y<a+\delta$ならば,
\begin{equation}
    \abs{\int^{x}_{\frac{x+y}{2}}dt\int^{t}_{\frac{x+y}{2}}ds\qty(f''(s)-f''\qty(\frac{x+y}{2}))}\leq 2\varepsilon\cdot \int^{x}_{\frac{x+y}{2}}dt\int^{t}_{\frac{x+y}{2}}ds=\frac{(x-y)^2\varepsilon}{2}
\end{equation}
など.
\subsection{2019年度}
問題は
\begin{quote}
    \url{https://www.math.titech.ac.jp/top/~jimu/Graduate/old-exam/H31innsi.pdf}
\end{quote}
を見よ.
\subsubsection*{[1]}
(1) $A^{12}=(A^3)^4=O$である.

(2) $A$の最小多項式,固有多項式をそれぞれ$\varphi(X),\ \Phi(X)$とおく.(1)より$\varphi\mid X^{12}$であるから$\varphi=X^k$の形である.$\Phi\mid \varphi^4$であるから$\Phi=X^4$となる.Cayley--Hamiltonの定理より$\varphi\mid \Phi=X^4$.これは$A^4=O$であることを示している.

(3) 
\begin{equation}
    O=A^6=(A^3)^2=\begin{bmatrix}
        0 & a & d & f \\
          & 0 & b & e \\
          &   & 0 & c \\
          &   &   & 0 
    \end{bmatrix}^2=\begin{bmatrix}
        0 & 0 & ab & ae+cd \\
          & 0 & 0  & bc \\
          &   & 0  & 0 \\
          &   &    & 0 
    \end{bmatrix}
\end{equation}
より$ab=0$である.

\subsubsection*{[2]}\label{am_2019_2}
(1) 固有値は$0,\ 2,\ -2$で,固有空間の次元はそれぞれ$3,\ 2,\ 1$.

(2) まず$A,\ B$が共通の固有値$\lambda$を持つと仮定する.零でない$x\in \Cset^m,\ y\in \Cset^n$があり,
\begin{equation}
    Ax=\lambda x,\quad \transpose{B}y=\lambda y    
\end{equation}
となる.すると$X=x\transpose{y}\in M(m,n)$は零ではなく
\begin{equation}
    AX-XB=Ax\transpose{y}-x\transpose{y}B=\lambda x\transpose{y}-x\transpose{(\lambda y)}=0
\end{equation}
となる.

次に$A$と$B$が共通の固有値を持たないと仮定する.このとき$A,\ B$の最小多項式$\varphi(T),\ \psi(T)\in \Cset[T]$は互いに素であるから,$u,v\in \Cset[T]$があり
\begin{equation}
    u\varphi+v\psi=1
\end{equation}
となる.さて,変数への$A$の代入によって多項式環$\Cset[T]$は$M(m,n)$に作用している.つまり
\begin{equation}
    \Cset[T]\times M(m,n)\ni(p,\ X)\mapsto p(A)X\in M(m,n)
\end{equation}
という作用がある.もし$X\in \ker f_{A,B}$,つまり$AX=XB$ならば任意の$p\in \Cset[T]$に対して
\begin{equation}
    p(A)X=X p(B)
\end{equation}
であるから,とくに
\begin{equation}
    X=EX=(u(A)\varphi(A)+v(A)\psi(A))X=Xv(B)\psi(B)=0 
\end{equation}
となる.これは$\ker f_{A,B}=0$であることを示している.
\subsubsection*{[3]}
(1) $(0,0),\ (0,1)\in\Rset^2$は$\mathcal{O}'$の開集合で分離できない.

(2) $A\subset\Rset^2,\ (0,0)\notin A$のとき$\interior_{\mathcal{O}'}(A)=\emptyset$.

(3) $A\subset \Rset^2,\ (0,0)\in A$のとき$\closure_{\mathcal{O}'}(A)=\Rset^2$.

(4) $(\Rset^2,\ \mathcal{O}')$は連結である.$X=U\cup V,\ U,V\in\mathcal{O}',',\ U\cap V=\emptyset$とする.一般性を失わず$(0,0)\in U$としてよい.すると$(0,0)\notin V$より$V=\emptyset$である.

(5) $f\colon (\Rset^2,\mathcal{O}')\to(\Rset^2,\mathcal{O}')$を同相写像とする.$f(0,0)\neq(0,0)$だと仮定しよう.$\varepsilon>0$で$f(0,0)\notin B_\varepsilon=\qty{p\in \Rset^2\mid \abs{p}<\varepsilon}$となるものを取る.$f$は連続で$B_{\varepsilon}\in\mathcal{O}'$より$f^{-1}(B_{\varepsilon})\in\mathcal{O}'$であり,かつ$(0,0)\notin f^{-1}(B_{\varepsilon})$であるから$f^{-1}(B_{\varepsilon})=\emptyset$となる.これは$f$が全単射であることに反する.

\subsubsection*{[4]}
(1) $F\colon D\to \widetilde{D},\ F(u,v)=(x(u,v),\ y(u,v))$は$F\circ F^{-1}=\mathrm{id}_{\widetilde{D}},\ F^{-1}\circ F=\mathrm{id}_{D}$を満たすので,chain ruleよりJacobi行列$JF$について$(JF)(JF^{-1})=I$が成り立つ.行列式を取って結論を得る.

(2) $\widetilde{E}=[1,2]\times [-2,2]$とし$F\colon E\to \widetilde{E}$を$F(x,y)=(u(x,y),v(x,y))=(xy,y-x^2)$で定める.
\begin{equation}
    \frac{\partial(u,v)}{\partial(x,y)}=\det\begin{bmatrix}
        y & x \\
        -2x & 1 
    \end{bmatrix}=y+2x^2\neq0\quad \text{in $E$}
\end{equation}
である.$F$は全単射かつ$C^1$級で$F^{-1}$も$C^1$級(逆関数定理より).よって
\begin{align}
    \iint_{E}\qty(\frac{1}{x}+\frac{2x}{y})\,dxdy&=\int_1^2du\int_{-2}^{2}dv\qty(\frac{1}{x}+\frac{2x}{y})\frac{\partial(x,y)}{\partial(u,v)}\\
    &=\int_1^2du\int_{-2}^{2}\qty(\frac{1}{x}+\frac{2x}{y})\frac{\partial(x,y)}{\partial(u,v)}\\
    &=\int_{1}^2du\int_{-2}^2dv\frac{1}{u}\\
    &=4\log2
\end{align}

\subsubsection*{[5]}
(1) 帰納的に$0\leq f_n(x)\leq 2,\ (n\in\Nset,\ x\in I)$がわかる.よって
\begin{align}
    f_{n+1}(x)-f_n(x)&=\sqrt{f_n(x)+2}-f_n(x)\\
    &=\frac{f_n(x)+2-(f_n(x))^2}{\sqrt{f_n(x)+2}+f_n(x)}\\
    &\geq \frac{1}{4}(f_n(x)+1)(2-f_n(x))\\
    &\geq0 
\end{align}
である.

(2) 
\begin{align}
    2-f_{n+1}(x)&=2-\sqrt{f_n(x)+2}\\
    &=\frac{2-f_n(x)}{2+\sqrt{f_n(x)+2}}\\
    &\leq \frac{1}{2}(2-f_n(x)) 
\end{align}
である.よって
\begin{equation}
    0\leq 2-f_n(x)\leq \qty(\frac{1}{2})^{n-1}(2-f_1(x))\leq \qty(\frac{1}{2})^{n-1}\cdot 2\to0 \ (n\to\infty)
\end{equation}

(3)\ $f$は定数関数$2$に$[0,1]$上一様収束する.よって2.
\subsection{2020年度}
問題は
\begin{quote}
    \url{https://www.math.titech.ac.jp/top/~jimu/Graduate/old-exam/2020innsi.pdf}
\end{quote}
を見よ.
\subsubsection*{[1]}
はい.
\begin{align}
    \abs{\int_{a}^{b}f(x)\,dx-\frac{b-a}{N}\sum_{k=1}^N f(x_k)}
    &=\abs{\sum_{k=1}^N \int_{x_{k-1}}^{x_k}f(x)\,dx-\sum_{k=1}^N \int_{x_{k-1}}^{x_k} f(x_k)\,dx}\\
    &\leq \sum_{k=1}^N \int_{x_{k-1}}^{x_k}\abs{f(x)-f(x_k)}\,dx\\
    &\leq \sum_{k=1}^N \int_{x_{k-1}}^{x_k}dx\,\abs{\int_{0}^{1}\frac{d}{dt}f(tx+(1-t)x_k)\,dt}\\
    &\leq \sum_{k=1}^N \int_{x_{k-1}}^{x_k}dx\int_{0}^{1}dt\,\abs{x-x_k}\abs{f'(tx+(1-t)x_k)}\\
    &\leq \sum_{k=1}^N \int_{x_{k-1}}^{x_k}dx\,K\frac{b-a}{N}\\
    &=N\cdot\frac{b-a}{N}\cdot K\frac{b-a}{N}=\frac{K(b-a)^2}{N}
\end{align}
追記:$f$は$C^1$級ではないので議論を修正しなければならないが,平均値の定理を使えばなんとでもなる.
\subsubsection*{[2]}
$n,m\in\Nset,\ m<n$に対して
\begin{align}
    \abs{\frac{a_n}{n}-\alpha}
    &= \abs{\frac{1}{n}\qty(a_m+\sum_{k=m}^{n-1}(a_{k+1}-a_k))-\alpha}\\
    &=\abs{\frac{a_m-m\alpha}{n}+\frac{1}{n}\sum_{k=m}^{n-1}(a_{k+1}-a_k-\alpha)}\\
    &\leq \frac{\abs{a_m}+m\abs{\alpha}}{n}+\frac{1}{n}\sum_{k=m}^{n-1}\abs{a_{k+1}-a_k-\alpha}\\
    &\leq \frac{\abs{a_m}+m\abs{\alpha}}{n}+\frac{n-m}{n}\sup_{k\geq m}\abs{a_{k+1}-a_k-\alpha}
\end{align}
となる.$\limsup_{n\to\infty}$をとって
\begin{equation}
    \limsup_{n\to\infty}\abs{\frac{a_n}{n}-\alpha}\leq \sup_{k\geq m}\abs{a_{k+1}-a_k-\alpha}
\end{equation}
である.$m\to\infty$として結論を得る.

(2) 答えは$\beta/2$.$n,m\in\Nset,\ m<n$に対して
\begin{align}
    \abs{\frac{a_n}{n^2}-\frac{\beta}{2}}
    &=\abs{\frac{\beta}{2}\qty(\qty(1-\frac{m}{n})\qty(1+\frac{m-1}{n})-1)+\frac{1}{n^2}\sum_{k=m}^{n-1}(a_{k+1}-a_{k}-k\beta)}\\
    &\leq \frac{\abs{\beta}}{2}\abs{\qty(1-\frac{m}{n})\qty(1+\frac{m-1}{n})-1}+\frac{1}{n}\sum_{k=m}^{n-1}\abs{\frac{a_{k+1}-a_{k}}{k}-\beta} 
\end{align}
あとは同様.

\subsubsection*{[3]}
(1) 答えは$\qty{(0,0)}$.$Y\subset X$を,$\qty{(0,0)}\subsetneq Y$となるものとする.もし$Y=\qty{(0,0),\ (0,1)}$ならば,$Y$の開集合$U=\qty{(0,0)},\ V=\qty{(0,1)}$が$Y$を分割する.もし$Y\neq \qty{(0,0),\ (0,1)}$の場合,$p=(1/n_0,y_0)\in Y,\ n_0\in\Nset,\ 0\leq y_0\leq 1$を取る.
\begin{equation}
    U=\qty{(1/n_0,y)\mid 0\leq y\leq 1},\quad V=Y\setminus U 
\end{equation}
が$Y$を分割する$Y$の開集合である.

(2) ない.

\subsubsection*{[4]}
(a)ならば(b)は明らか.(b)を仮定する.$f$が同型でも零写像でもないと仮定する.すると,$0\neq x\in V,\ \alpha\in V^\ast$があり,
\begin{equation}
    f(x)=0,\quad f^\ast(\alpha)\neq0 
\end{equation}
となる.ただし$V^\ast$は$V$の双対空間であり,$f^\ast\in \mathop{\mathrm{End}}(V^\ast)$は$f$の双対写像.$g\in \mathop{\mathrm{End}}(V)$を次で定める:
\begin{equation}
    g(v)=\alpha(v)x
\end{equation}
このとき
\begin{equation}
    f(g(v))=\alpha(v)f(x)=0,\quad g(f(v))=\alpha(f(v))x=f^\ast(\alpha)(v)x\quad (v\in V)
\end{equation}
だが,(b)の仮定より任意の$v\in V$に対して$f^\ast(\alpha)(v)x=0$したがって$f^\ast(\alpha)=0$となる.よって矛盾.
\subsubsection*{[5]}
(1) 固有値は$\pm 1$.$\lambda=\pm1$に対応する固有空間を$W_{\lambda}$と書くことにすれば
\begin{equation}
    W_{1}=\langle1,\ (-2x+b)^2\rangle ,\quad W_{-1}=\langle-2x+b,\ (-2x+b)^3\rangle
\end{equation}

(2) $\varphi_{a,b}$の固有値全体の集合は$\qty{a^k\mid 0\leq k\leq d}$.

(3) $\varphi_{a,b}$が対角化可能であるための条件は,「$a\neq 1$または$b=0$」.これを示す.

$a\neq 1$ならば,$((a-1)x+b)^k\ (k=0,1,\ldots,d-1)$は固有値$a^k$に属する固有ベクトルであり,すべて一次独立である.

$b=0$ならば,$x^k\ (k=0,1,\ldots,d-1)$は固有値$a^k$に属する固有ベクトルであり,すべて一次独立である.

$a=1$かつ$b\neq0$の場合,
\begin{equation}
    f_k(x)=\frac{1}{k!}\frac{x}{b}\qty(\frac{x}{b}-1)\cdots\qty(\frac{x}{b}-k+1)\quad (k\geq0)
\end{equation}
とおくと$f_k\ (0\leq k\leq d-1)$は一次独立であり,
\begin{align}
    \varphi_{1,b}f_{k}(x)&=\frac{1}{k!}\qty(\frac{x}{b}+1)\frac{x}{b}\cdots\qty(\frac{x}{b}-k+2)\\
    &=\frac{1}{k!}\frac{x}{b}\qty(\frac{x}{b}-1)\cdots\qty(\frac{x}{b}-k+1)+\frac{1}{(k-1)!}\frac{x}{b}\cdots\qty(\frac{x}{b}-k+2)\\
    &=f_{k}(x)+f_{k-1}(x)
\end{align}
となる.よってこの基底によって$\varphi_{1,b}$はJordan細胞
\begin{equation}
    \begin{bmatrix}
        1 & 1 & 0 & \cdots & 0 & 0 \\
        0 & 1 & 1 & \cdots & 0 & 0 \\
        \vdots & \vdots & \vdots & \ddots & \vdots & \vdots \\
        0 & 0 & 0 & \cdots & 1 & 1 \\
        0 & 0 & 0 & \cdots & 0 & 1 \\
    \end{bmatrix}
\end{equation}
で表現される.よって対角化可能ではない.

\subsection{2021年度}
問題は
\begin{quote}
    \url{https://www.math.titech.ac.jp/top/~jimu/Graduate/old-exam/2021innsi.pdf}
\end{quote}
を見よ.
\subsubsection*{[1]}
(1) $D=D_1$は
\begin{equation}
    D(e^x)=e^x,\quad D(xe^x)=e^x+xe^x,\quad e^{2x}=2e^{2x} 
\end{equation}
を満たすので,$D,\ D_n=D^n$の行列表示は
\begin{equation}
    D=\begin{bmatrix}
        1 & 1 & 0 \\
        0 & 1 & 0 \\
        0 & 0 & 2 
    \end{bmatrix},\quad D_n=\begin{bmatrix}
        1 & n & 0 \\
        0 & 1 & 0 \\
        0 & 0 & 2^n  
    \end{bmatrix}\quad (n\geq0)
\end{equation}
である.$\qty{D_i}_{0\leq i\leq 2}$が一次独立であることを示そう:$a_i\in\Rset,\ a_0D_0+a_1D_1+a_2D_2=0$は
\begin{equation}
    a_0+a_1+a_2=a_1+2a_2=a_0+2a_1+4a_2=0 
\end{equation}
と同値でこれは$a_0=a_1=a_2=0$と同値.

(2) $n\geq 3$に対して
\begin{equation}
    D_n=(2^n-2n)D_0+(-2^{n+1}+3n+2)D_1+(2^n-n-1)D_2 
\end{equation}
であるから$\qty{D_i}_{0\leq i\leq 2}$が
$\mathop{\mathrm{span}}\qty{D_n\mid n\geq0}$の基底を与える.

(3)\ 普通に
\begin{equation}
    f(x)=\qty(\frac{1}{2}-\frac{n}{4})e^x+\frac{1}{2}xe^x+\frac{1}{1+2^n}e^{2x}
\end{equation}
\subsubsection*{[2]}
(1) $A_n=\qty{n,n+1,n+2,\ldots}$.(2) 閉集合の族$\qty{A_{\lambda}}_{\lambda\in\Lambda}$が($\ast$)を満たすことは,開集合たち$U_\lambda=X\setminus A_\lambda$が「$\emptyset\neq F\subset \Lambda$が有限ならば$\bigcup_{\lambda\in F}U_{\lambda}\neq X$.かつ$\bigcup_{\lambda\in\Lambda}U_{\lambda}=X$.」このことから明らか.(3) $X$をコンパクト空間,$Y\subset X$をその閉集合とする.$Y$の任意の閉集合の族は$\qty{A_{\lambda}\cap Y}_{\lambda\in\Lambda},\ A_{\lambda}$は$X$の閉集合という形をしている.今$Y$は$X$の閉集合より$A_{\lambda}\cap Y$もそう.よって$Y$の閉集合族は($\ast$)を満たし得ない.
\subsubsection*{[3]}
(1) $(a,b)=(2,0)$.(2) $(a,b)=(4,0)$.(3) $(a,b)=(3,0)$.
\subsection{2022年度}
問題は
\begin{quote}
    \url{https://www.math.titech.ac.jp/top/~jimu/Graduate/old-exam/2022innsi.pdf}
\end{quote}
を見よ.
\subsubsection*{[1]}
固有多項式は
\begin{equation}
    (\lambda-1)^2(\lambda+1)(\lambda^2+\lambda+1)
\end{equation}
である.

固有値は$1,\ -1,\ (-1+\sqrt{-3})/2\eqcolon \omega_{\pm}$である.各固有空間$V_{\lambda}$は
\begin{equation}
    V_{1}=\langle\bm{v_1}+\bm{v_3}+\bm{v_5},\ \bm{v_2}+\bm{v_4}\rangle,\quad V_{-1}=\langle\bm{v_2}-\bm{v_4}\rangle,\quad V_{\omega_{\pm}}=\langle\bm{v_1}+\omega_{\pm}\bm{v_3}+\omega_{\mp}\bm{v_5}\rangle
\end{equation}
という基底を持つ.よって$f_\sigma$の最小多項式は
\begin{equation}
    (\lambda-1)(\lambda+1)(\lambda^2+\lambda+1)
\end{equation}
と求まる.
\subsubsection*{[2]}
$\ker(f),\ \mathop{\mathrm{Im}}(f)$の$V$における補空間$W,W'$を取る.つまり$V=\ker(f)+ W=\mathop{\mathrm{Im}}(f)+W'$かつ$\ker (f) \cap W=\mathop{\mathrm{Im}}(f)\cap W'=0$.次の可換図式を満たす$g_1,h$を取る:
\begin{equation}
    % https://tikzcd.yichuanshen.de/#N4Igdg9gJgpgziAXAbVABwnAlgFyxMJZARgBoAGAXVJADcBDAGwFcYkQA1EAX1PU1z5CKAEwVqdJq3Zde-bHgJEyIiQxZtEIAOo8+IDAqFFypVTXXSts-YcFLRZtVM0gAOm4C29HAAsIaMAe3n4ATp7AAJKe3NwAFABmAJQ8EjBQAObwRKAJoRCeSKYgOBBIYpIa7ME+-oE1YRFoobEA+sDa3CA0jPQARjCMAAoCisIgoVgZvjh6ufmFiMWlSADMFi7svt0gvQPDo8Zak9OzciB5BeU0K4jrJfRYjFsQEADWcxcL1yVliAAsGyqWg8AGMCBlPpdFoDfkgyA8ni93lDvksbn8EZZXAlUVdEBVbljNloMq1iDtGFgwK4oPQ4L50qluEA
    \begin{tikzcd}
      & V \arrow[dd, "\mathop{\mathrm{pr}}_{W}"'] \arrow[ldd, "h"'] \arrow[r, "f"] & V                                        \\
      &                                                                            &                                          \\
    V & W \arrow[l, hook] \arrow[r, "\cong"] \arrow[ruu, "g_1", dashed]            & \mathop{\mathrm{Im}}(f) \arrow[uu, hook]
    \end{tikzcd}
\end{equation}
さらに同型$g_2\colon \ker(f)\to W'$がある.$g=g_1\oplus g_2\colon \ker(f)\oplus W$とおけばよい.
\subsubsection*{[3]}\label{am_2022_3}
(1) いいえ.(2) はい.(3) はい(多分.なぜ?).(4) いいえ.

\subsubsection*{[4]}
(1) $f_{n}|_{[0,1]}$は$[0,1]$上の一様有界かつ同程度一様連続な関数列である.よってAscoli--Arzelàの定理より$f_n$は$[0,1]$上一様収束する部分列を持つが,$f$は周期1を持つので$\Rset$上一様収束する.

(2) 同様にAscoli--Arzelàの定理より.

\subsubsection*{[5]}
(1) $\pi/t$.

(2) 答えは0.$x\in[a,b]$に対して$tf(x)/(t^2+x^2)\to0\ (t\to0)$であり
\begin{equation}
    \abs{\frac{t}{t^2+x^2}f(x)}\leq \norm{f}_{L^\infty([a,b])}\frac{1}{x^2} \in L^1([a,b])\quad (0<t<1) 
\end{equation}
である($0\notin [a,b]$に注意せよ.).dominated convergenceより問題の積分は0に収束する.

(3) $\widetilde{f}=\begin{cases}
    f & (\text{on $[a,b]$})\\
    0 & (\text{outside $[a,b]$})
\end{cases}$とおく.十分小さな$\delta>0$に対して
\begin{align}
    \abs{\pi f(0)-\int_{a}^b \frac{tf(x)}{t^2+x^2}\,dx}
    &=\abs{\int_{-\infty}^{\infty}\frac{t(\widetilde{f}(0)-\widetilde{f}(x))}{t^2+x^2}\,dx}\\
    &\leq \int_{-\delta}^{\delta}\frac{t}{t^2+x^2}\abs{f(0)-f(x)}\,dx + \int_{\Rset\setminus(-\delta,\delta)}\frac{t}{t^2+x^2}\abs{\widetilde{f}(0)-\widetilde{f}(x)}\,dx\\
    &\leq \pi\sup_{x\in (-\delta,\delta)}\abs{f(0)-f(x)}+2\norm{f}_{L^\infty([a,b])}\qty(\pi-2\arctan\frac{\delta}{t})
\end{align}
となる.$\limsup_{t\to+0}$をとって
\begin{equation}
    \limsup_{t\to +0}\abs{\pi f(0)-\int_{a}^b \frac{tf(x)}{t^2+x^2}\,dx}\leq \pi\sup_{x\in(-\delta,\delta)}\abs{f(0)-f(x)}
\end{equation}
を得る.$\delta\to0$として結論を得る.
\subsection{2023年度}
問題は
\begin{quote}
    \url{https://www.math.titech.ac.jp/top/~jimu/Graduate/old-exam/2023innsi.pdf}
\end{quote}
を見よ.
\subsubsection*{[1]}
(1) $W\in \mathcal{S}$とせよ.このとき任意の$n\in\Nset$に対して$W=\varphi^n(W)\subset \varphi(V)$なので$W\subset W_0$を得る.

(2) $V\supset \varphi(V)\supset \varphi^2(V)\supset\cdots$という部分空間の系列を考える.$V$は有限次元なので,ある$N\in\Nset$に対して$\varphi^N(V)=\varphi^{N+1}(V)$となる.このとき$\varphi^{n}=\varphi^{n+1}(V)\ (n\geq N)$となるので$W_0=\varphi^N(V)$である.よって$\varphi(W_0)=\varphi^{N+1}(V)=W_0$であり,$\varphi|_{W_0}\colon W_0\to W_0$は全射なので(有限次元性より)同型.

(3) $K=\Rset,\ V=\Rset^{\oplus\Nset},\ \varphi((a_1,a_2,\ldots))=(a_2,a_3,\ldots)$.

\subsubsection*{[2]}
(1) $D_n(x;a,b)=xD_{n-1}(x;a,b)-abD_{n-2}(x;a,b)$がわかる.とくに$a_n=D_n(-2;1,1)$とおくと$a_n=-2a_{n-1}-a_{n-2}$.$a_1=-2,\ a_2=3$とあわせて$a_n=(-1)^{n}(n+1)$.

(2) 帰納法.

(3) $D_n(x;a,b)$は行列
\begin{equation}
    \begin{bmatrix}
        0 & a & 0 & \cdots & 0 \\
        b & 0 & a & \cdots & 0 \\
        0 & b & 0 & \cdots & 0 \\
        \vdots & \vdots & \vdots & \ddots & \vdots \\
        0 & 0 & 0 & \cdots & 0
    \end{bmatrix}
\end{equation}
の固有多項式である.一方この行列は固有値$2\sqrt{ab}\cos(k\pi/(n+1))\ (k=1,2,\ldots,n)$を持ち,対応する固有ベクトルは
\begin{equation}
    \qty[\sqrt{a^{m-1}b}\sin\frac{k\pi}{n+1},\ \sqrt{a^{m-2}b^2}\sin\frac{2\pi}{n+1},\ \cdots,\ \sqrt{ab^{m-1}}\sin \frac{n\pi}{n+1}]
\end{equation}
である.
\subsubsection*{[3]}\label{am_2023_3}
(1) いいえ.(2) はい.(3)の答えは
\begin{equation}
    \overline{B}=\qty{(x,y)\in\Rset^2\mid \text{($x\leq 1/\sqrt{2}$かつ$y\leq 1/\sqrt{2}$)または($x\geq -1/\sqrt{2}$かつ$y\geq -1/\sqrt{2}$)}}
\end{equation}
(4) 謎.

\subsubsection*{[4]}
(1) $1<a<2$のとき,($\abs{\sin x}\leq \abs{x}$に注意して)
\begin{equation}
    \int_{0}^\infty \abs{\frac{\sin x}{x^{\alpha}}}\,dx\leq \int_{0}^{1}x^{1-\alpha}\,dx+\int_{1}^\infty x^{-\alpha}\,dx\leq \frac{1}{2-\alpha}+\frac{1}{1-\alpha}<\infty 
\end{equation}
より従う.

(2) $0<a\leq 1$のとき,$1<M,N$に対して部分積分より
\begin{align}
    \abs{\int_M^N \frac{\sin x}{x^\alpha}\,dx}
    &=\abs{\frac{\cos M}{M^\alpha}-\frac{\cos N}{N^\alpha}-\alpha\int_{M}^{N}x^{-\alpha-1}\cos x\,dx}\\
    &\leq \frac{2}{M^\alpha}+\frac{2}{N^\alpha}\\
    &\to 0 \quad (M,N\to\infty)
\end{align}
なので$\int_{0}^1(\sin x)/x^\alpha\,dx$は収束する.$\int_0^1(\sin x)/x^\alpha\,dx$は(1)と同様に絶対収束する.

(3) 
\begin{equation}
    \int_{1}^\infty \abs{\frac{\sin x}{x^\alpha}}\,dx=\sum_{k=1}^\infty \int_{k\pi}^{(k+1)\pi}\frac{\abs{\sin x}}{x^\alpha}\,dx\geq \sum_{k=1}^\infty \frac{1}{(k\pi)^\alpha}\int_{k\pi}^{(k+1)\pi}\abs{\sin x}\,dx=\sum_{k=1}^\infty \frac{2}{(k\pi)^\alpha}=\infty
\end{equation}
\subsubsection*{[5]}
(1) $n\geq 2$に対して$1/(nk)^{nk}\leq 1/2^n$である.

(2) $I_{n,k}=\int_{0}^1 x^k (\log x)^n\,dx$とおく.部分積分より
\begin{equation}
    I_{n,k}=\int_{0}^1\qty(\frac{x^{k+1}}{k+1})'(\log x)^n=-\int_{0}^1\frac{x^{k+1}}{k+1}\cdot n(\log x)^{n-1}\frac{1}{x}\,dx=-\frac{n}{k+1}I_{n-1,k}
\end{equation}
よって 
\begin{equation}
    I_{n,n}=(-1)^n\frac{n!}{(n+1)^n}I_{0,n}=(-1)^n\frac{n!}{(n+1)^{n+1}}
\end{equation}
である.

(3) 
\begin{align}
    \int_{0}^{1}x^{x}\,dx&=\int_0^1 e^{x\log x}\,dx\\
    &=\int_{0}^1 \sum_{n\geq 0}\frac{(x\log x)^n}{n!}\\
    &=\sum_{n\geq0}\frac{1}{n!}\int_{0}^1 (x\log x)^n\,dx\\
    &=\sum_{n\geq 0}\frac{(-1)^n}{(n+1)^{n+1}}\\
    &=S(1)-2S(2)
\end{align}
\subsection{2024年度}
問題は
\begin{quote}
    \url{https://www.math.titech.ac.jp/top/~jimu/Graduate/old-exam/2024innsi.pdf}
\end{quote}
を見よ.
\subsubsection*{[1]}
(1) 
\begin{align}
    A^2&=\begin{bmatrix}
        1+(n-1)a^2 & 2a+(n-2)a^2 & \cdots & 2a+(n-2)a^2 \\ 
        2a+(n-2)a^2 & 1+(n-1)a^2 & \cdots & 2a+(n-2)a^2 \\
        \vdots & \vdots & \ddots & \vdots \\
        2a+(n-2)a^2 & 2a+(n-2)a^2 & \cdots & 1+(n-1)a^2
    \end{bmatrix}\\
    &=(na-2(n-1))A+(a-1)(na-(n-1))I
\end{align}
がわかる.$A$はスカラー行列ではないので最小多項式$\varphi$は1次式ではなく,
\begin{equation}
    \varphi(x)=x^2-(na-2(a-1))x-(a-1)(na-(a-1))
\end{equation}
がわかる.

(2) $\varphi(x)=(x+a-1)(x-na+a-1)$より$A$の固有値は$-a+1,\ na-a+1$.

(3) $a$が1とも$-1/(n-1)$とも異なる場合,$A$は正則なので$\rank A=n$.

$a=1$の場合,簡単に$\rank A=1$.

$a=-1/(n-1)$の場合,$A$を基本変形を施して
\begin{equation}
    \begin{bmatrix}
        0 & 0 & 0 & \cdots & 0 \\
        0 & 1 & -1/(n-1) & \cdots & -1/(n-1) \\
        0 & -q/(n-1) & 1 & \cdots & -1/(n-1) \\
        \vdots & \vdots & \vdots & \ddots & \vdots \\
        0 & -1/(n-1) & -1/(n-1) & \cdots & 1
    \end{bmatrix}
\end{equation}
とできるが,右下の行列は一番最初の場合より$\rank=n-1$である.よって$\rank=n-1$.

\subsubsection*{[2]}
(1) $f\colon M_n(\Rset)\to\Rset$を線形写像とする.このとき行列$A=(A_{ij})_{i,j}$を$a_{ij}=f(E_{ji})$で定める.ここで$E_{ij}$は$(i,j)$-成分だけ1で他が0の行列を表す.$f(X)=\tr(AX)$であることを示そう.線形写像は基底の行き先で決まるので$X=E_{ij}$について示せばよい.
\begin{equation}
    \tr(AE_{ij})=\sum_{k,l}A_{kl}(E_{ij})_{lk}=A_{ji}=f(E_{ij})
\end{equation}
よりよい($E_{ij}$の$(l,k)$-成分を$(E_{ij})_{li}$で表す).

(2) 答えは$g(X)=c\tr(X),\ c\in\Rset$.$g(X)=\tr(CX),\ C\in M_n(\Rset)$が$g(AB)=g(BA)\ (\forall A,B\in M_n(\Rset))$を満たすとする.つまり
\begin{equation}
    \tr(CAB)=\tr(CBA)\quad (\forall A,B\in M_n(\Rset)).
\end{equation}
とくに$A=E_{st},\ B=E_{uv}$とおくと,
\begin{equation}
    \sum_{i,j,k}C_{ij}(E_{st})_{jk}(E_{uv})_{ki}=\sum_{i,j,k}C_{ij}(E_{uv})_{jk}(E_{st})_{ki}
\end{equation}
したがって$C_{vs}\delta_{tu}=C_{tu}\delta_{vs}$,つまり
\begin{equation}
    v\neq s\Longrightarrow C_{vs}=0;\quad v=s,\ t=u\Longrightarrow C_{uu}=C_{vv}
\end{equation}
である.これは$C$がスカラー行列であることを示している.

(3) 答えは$n^2-1$次元.包含写像$\iota\colon V\to M_n(\Rset)$の双対写像
\begin{equation}
    \iota^\ast\colon M_n(\Rset)^\ast\to V^\ast,\quad \iota^\ast(f)=f\circ \iota=f|_{V}  
\end{equation}
を考える.$\iota$は単射より$\iota^\ast$は全射.また(2)より$\ker\iota^\ast$は$\tr$で生成されるので1次元.よって準同型定理より$V^\ast$は$n^2-1$次元である.

\subsubsection*{[3]}
(1) $X\ni x\mapsto(x,f(x))\in Y$は連続なので像を制限しても$X\to \Gamma(f)$は連続である.連結空間の全射連続写像の像として$\Gamma(f)$は連結.

(2) $x\in f^{-1}(A)$とする.このとき$(x,f(x))\in p_Y^{-1}(A)\cap\Gamma(f)$より,$x=p_X(x,f(x))\in p_X(p_Y^{-1}(A)\cap \Gamma(f))$.

逆に$x\in p_X(p_Y^{-1}(A)\cap \Gamma(f))$とする.このとき,ある$y\in Y$があり$(x,y)\in p_{Y}^{-1}(A)\cap \Gamma(f)$.よって$f(x)=y\in A$となる.

(3) $C\subset X\times Y$を閉集合とする.このとき$p_X(C)$が$X$の閉集合であることを示す.$x\in X\setminus p_X(C)$とする.つまり任意の$y\in Y$に対して$(x,y)\notin C$.$C$は$X\times Y$の閉集合なので,$X,Y$それぞれの開集合$U_y\subset X,\ V_y\subset Y$があり,
\begin{equation}
    (x,y)\in U_y\times V_y\subset (X\times Y)\setminus C 
\end{equation}
となる.$\qty{V_y}_{y\in Y}$は$Y$の開被覆なので,コンパクト性より有限部分被覆$V_{y_1},\ldots,V_{y_n}$が存在する.
\begin{equation}
    U\coloneq U_{y_1}\cap\cdots\cap U_{y_n} 
\end{equation}
は$X$の開集合であり,$U\subset X\setminus p_X(C)$である:$x\in U$を固定する.任意の$y\in Y$について,ある$i$に対して$y\in V_{y_i}$となる.$(x,y)\in U_{y_i}\cap V_{y_i}\subset (X\times Y)\setminus C$となり,$y\in Y$は任意なので$x\notin p_X(C)$.

(4) $f$が連続だとする.$(x,y)\in (X\times Y)\setminus \Gamma(f)$だとする.このとき$y\neq f(x)$である.$Y$はHausdorff空間なので,開集合$U,V\subset Y$があり,$f(x)\in U,\ y\in V,\ U\cap V=\emptyset$となる.$f$は連続より$f^{-1}(U)\times V$は$X\times Y$の開集合であり,$f^{-1}(U)\times V\subset (X\times Y)\setminus \Gamma(f)$となる.

逆に$\Gamma(f)$が閉だとする.このとき任意の閉集合$A\subset Y$に対して
\begin{equation}
    f^{-1}(A)=p_X(p_{Y}^{-1}(A)\cap \Gamma(f)) 
\end{equation}
は,(2), (3)より$X$の閉集合である.

\subsubsection*{[4]}
答えは$\beta>1,\ \alpha+\beta>2$.

(i) 問題の級数が収束する条件は$\alpha+\beta>1$.

(ii) このとき,級数$\sum_{n\geq 1}n^{-\alpha}(x+n)^{-\beta}$は$f(x)$に広義一様収束する.よって$f$は連続.

(iii) 広義一様収束なので,任意の$R>0$に対して
\begin{equation}
    \int_{0}^{R}f(x)\,dx=\sum_{n\geq 1}\int_{0}^{R}\frac{1}{n^{\alpha}(x+n)^{\beta}}
\end{equation}
である.$\beta=1$の場合は
\begin{equation}
    \int_{0}^{R}f(x)\,dx=\sum_{n\geq 1}\frac{1}{n^\alpha}\log\qty(1+\frac{R}{n})\to\infty\ (R\to\infty)
\end{equation}
より広義積分は発散する.$\beta\neq 1$の場合
\begin{equation}
    \int_{0}^{R}f(x)\,dx=\sum_{n\geq 1}\frac{1}{(\beta-1)n^{\alpha}}\qty(\frac{1}{n^{\beta-1}}-\frac{1}{(R+n)^{\beta-1}})
\end{equation}
が$R\to\infty$で収束する条件は$\beta-1>0$かつ$\alpha+\beta-1>1$.
\subsubsection*{[5]}
(1) $\sqrt{a_na_{n+1}}\leq \frac{a_n+a_{n+1}}{2}$より.$a_n=1+2^{-n}+(-1)^{n-1}(1-2^{-n})$を考えれば逆は成り立たない.

(2) $\varepsilon=1$に対して$\delta>0$があり,
\begin{equation}
    x,y\in [0,\infty),\ \abs{x-y}\leq \delta\Longrightarrow \abs{f(x)-f(y)}\leq 1 
\end{equation}
となる.よって$x\geq \delta$ならば
\begin{equation}
    \int_{x-\delta}^{x+\delta}\abs{f(x)-f(y)}\,dy\leq 2\delta
\end{equation}
一方
\begin{equation}
    \int_{x-\delta}^{x+\delta}\abs{f(x)-f(y)}\,dy\geq \abs{\int_{x-\delta}^{x+\delta}(f(x)-f(y))\,dy}\geq2\delta\abs{f(x)}-\abs{\int_{x-\delta}^{x+\delta}f(y)\,dy}\geq 2\delta\abs{f(x)}-2\sup_{[0,\infty)}\abs{F}
\end{equation}
よって
\begin{equation}
    \sup_{x\in[\delta,\infty)}\abs{f}\leq 1+\frac{1}{\delta}\sup_{[0,\infty)}\abs{F}
\end{equation}
を得る.
\begin{equation}
    \sup_{x\in[0,\infty)}\abs{f}\leq 1+\frac{1}{\delta}\sup_{[0,\infty)}\abs{F}+\sup_{[0,\delta]}\abs{f}<\infty 
\end{equation}
である.

\section{午後}
主に解析系の問題を解いていく.
\subsection{2015年度}
\subsubsection*{[5]}
結論:$A_n\to -2$,$B_n$は収束しない.$f_n(x)=(nx^2\cos x)/(1+nx),\ g_n(x)=(nx\cos x)/(1+nx^2)$とおく.

$f_n\to x\cos x$ a.e.である.また$f_n(x)\leq x\in L^1([0,\pi])$.よってdominated convergenceが適用できて
\begin{equation}
    A_n=\int_{0}^{\pi}f_n(x)\,dx\to \int_{0}^{\pi}x\cos x\,dx=-2
\end{equation}
を得る.

$B_n=\int_{0}^{\pi}g_n(x)\,dx$が発散することを示す.$g_n\to (\cos x)/x$ a.e.に注意する.積分区間を$\int_{0}^{\pi/3}+\int_{\pi/3}^{\pi}$に分ける.$\abs{g_n(x)}\leq 1/x\leq 3/\pi\ (x\in[\pi/3,\pi])$より$\int_{\pi/3}^{\pi}g_n(x)\,dx$は収束する.一方
\begin{equation}
    \int_{0}^{\pi/3}g_n(x)\,dx\geq \int_{0}^{\pi/3}\frac{nx}{1+nx^2}\cdot\frac{1}{2}\,dx=\frac{1}{4}\log\qty(1+\frac{n\pi^2}{9})\to\infty 
\end{equation}
である.

\subsubsection*{[7]}
(1) 関数$1/(1+z^{2n+1})$の閉曲線$C_R=\partial D_R$上の線積分を考える.ただし
\begin{equation}
    D_R=\qty{re^{i\theta}\in \Cset\mid 0\leq r\leq R,\ 0\leq\theta\leq 2\pi/(2n+1)}.
\end{equation}
$D_R$における$1/(1+z^{2n+1})$の極は$\zeta=\exp(\pi i/(2n+1))$ただ1つで,$C_R$上にはない.留数定理より
\begin{align}
    \int_{C_R}\frac{dz}{1+z^{2n+1}}&=2\pi i\Residue(\zeta,1/(1+z^{2n+1}))\\
    &=2\pi i\lim_{z\to\zeta}\frac{z-\zeta}{1+z^{2n+1}}\\
    &=2\pi i\qty(\prod_{k=1}^{2n}(\zeta-\zeta^{2n+1}))^{-1}\\
    &=2\pi i\zeta^{-2n}\qty(\prod_{k=1}^{2n}(1-\zeta^{2k}))^{-1}
\end{align}
である.今,$z^{2n+1}-1$の根は$z=\exp(2\pi k/(2n+1))=\zeta^{2k}\ (k=0,1,\ldots,2n)$であるから因数分解公式
\begin{equation}
    (z-1)(z^{2n}+\cdots+z+1)=z^{2n+1}-1=\prod_{k=0}^{2n}(z-\zeta^{2k})
\end{equation}
が成り立つ.両辺を$z-1$で割って$z\to1$の極限を考えると
\begin{equation}
    2n+1=\prod_{k=1}^{2n}(1-\zeta^{2k})
\end{equation}
を得る.したがって線積分の結果は
\begin{equation}
    \int_{C_R}\frac{dz}{1+z^{2n+1}}=\frac{2\pi i\zeta^{-2n}}{2n+1}=-\frac{2\pi i\zeta}{2n+1}
\end{equation}
を得る.さて,ここで線積分の経路を$C_R=I_R\cup \Gamma_R\cup J_{R}$,
\begin{equation}
    I_{R}=[0,R],\quad \Gamma_R=\qty{Re^{i\theta}\mid 0\leq \theta\leq 2\pi/(2n+1)},\quad J_R=\qty{r\zeta^2\mid 0\leq r\leq R}
\end{equation}
と分ける.
\begin{equation}
    \int_{J_R}\frac{dz}{1+z^{2n+1}}=\int_{R}^{0}\frac{\zeta^2dr}{1+r^{2n+1}}=-\zeta^2\int_{0}^{R}\frac{dx}{1+x^{2n+1}}
\end{equation}
\begin{equation}
    \abs{\int_{\Gamma_R}\frac{dz}{1+z^{2n+1}}}\leq \frac{1}{R^{2n+1}-1}\cdot\frac{2\pi R}{n}\to 0\quad (R\to\infty)
\end{equation}
より,$R\to\infty$における極限で
\begin{equation}
    (1-\zeta^2)\int_{0}^{\infty}\frac{dx}{1+x^{2n+1}}=-\frac{2\pi i \zeta}{2n+1}
\end{equation}
が成り立つ.したがって
\begin{equation}
    \int_{0}^{\infty}\frac{dx}{1+x^{2n+1}}=\frac{\pi}{(2n+1)\qty(\zeta-\zeta^{-1})/(2i)}=\frac{\pi}{(2n+1)\sin(\pi/(2n+1))}
\end{equation}
が示された.

(2) $\log\abs{\varphi(z)}$が$z$に関する調和関数であることが計算により分かる.

\subsubsection*{[7]}
(1) $h_1(x,t)=\sin x+2t^{1/2}\sin^2 x$.

(2) $T<1/64$でよいことを帰納法で示す.$0<t<1/64$ならば$\abs{h_1(x,t)}\leq 1+2t^{1/2}\leq 2$なので$j=0,1$での成立はよい.$\norm{h_j}_{T}\leq 2$を仮定する.このとき$0<t<T$に対して
\begin{equation}
    \abs{h_{j+1}(x,t)}\leq \abs{h_0(x,t)}+\int_{0}^{t}\frac{\norm{h_j}_{T}^2}{(t-s)^{1/2}}\,ds\leq 1+2t^{1/2}\norm{h_j}_{T}^2\leq 1+2\cdot \frac{1}{64^{1/2}}\cdot 2^2\leq 2 
\end{equation}
を得る.よって$\norm{h_{j+1}}_{T}\leq 2$である.

(3) $T<1/256$でよい.実際,$0<t<T$に対して
\begin{align}
    \abs{h_{j+1}(x,t)-h_{j}(x,t)}&\leq \int_{0}^{t}\frac{\abs{h_{j}(x,t)}^2-\abs{h_{j-1}(x,t)}^2}{(t-s)^{1/2}}\,ds\\
    &\leq (\norm{h_j}_{T}+\norm{h_{j-1}}_{T})\norm{h_{j}-h_{j-1}}_{T}\cdot 2t^{1/2}\\
    &\leq (2+2)\norm{h_{j}-h_{j-1}}_{T}\cdot 2t^{1/2}\\
    &\leq \frac{1}{2}\norm{h_j-h_{j-1}}_{T}
\end{align}
となる.つまり$\norm{h_{j+1}-h_{j}}_{T}\leq \frac{1}{2}\norm{h_j-h_{j-1}}_{T}$.これを繰り返し用いて
\begin{equation}
    \norm{h_{j+1}-h_{j}}_{T}\leq \frac{1}{2}\norm{h_j-h_{j-1}}_{T}\leq \cdots\leq \frac{1}{2^{j}}(\norm{h_1}+\norm{h_0})\leq 2^{-j+2}
\end{equation}
を得る.よって,任意の$j<k$に対して
\begin{equation}
    \norm{h_j-h_{k}}_{T}\leq \sum_{i=j-1}^{k}\norm{h_{i+1}-h_{i}}_{T}\leq \sum_{i=j-1}^{k}2^{-i+2}=8(2^{-j}-2^{-k})\to0\quad (j,k\to\infty)
\end{equation}
である.
\subsection{2016年度}
\subsubsection*{[6]}
(1) $\infty>\int_{\Rset}f\geq \int_{\Rset\setminus F_\alpha}f\geq \alpha\abs{\Rset\setminus F_{\alpha}}$より従う.

(2) $x\in F_{\alpha}$ならば$g_n(x)\leq \frac{\alpha^n}{1+\alpha^n}\to 0$であり,$g_n\1_{F_{\alpha}}\leq f\in L^1$よりdominated convergenceが適用できる($0\leq y\leq 1$ならば$y^n/(1+y^n)\leq y$に注意せよ).

(3) $\int_{\Rset}=\int_{\qty{f>1}}+\int_{\qty{f<1}}+\int_{\qty{f=1}}$とわける.

$g_n\1_{\qty{f>1}}\to \1_{\qty{f>1}}$ a.e.かつ$g_n\1_{\qty{f>1}}\leq \1_{\qty{f>1}}\in L^1$ ((1)より)なので$\int_{\qty{f>1}}g_n\to \abs{\qty{f>1}}$を得る.

$f(x)<1$ならば$g_n(x)\1_{\qty{f<1}}(x)\leq (f(x))^n\to0$なので$g_n\1_{\qty{f<1}}\to0$ a.e., また$g_n\1_{\qty{f<1}}\leq f\in L^1$なので$\int_{\qty{f<1}}g_n\to0$を得る.

$g_n\1_{\qty{f=1}}=\frac{1}{2}\cdot\1_{\qty{f=1}}$より$\int_{\qty{f=1}}g_n=\abs{\qty{f=1}}/2$である.

\subsubsection*{[7]}
相図を描けば明らか.

\subsubsection*{[8]}
(1) 偏角の原理の証明問題.\hyperref[pm_2024_5]{2024年度問[5]}に同じ問題が出題されている.

(2) $D_{\pm}=\qty{z\in\Cset\mid \abs{z\pm i/2}\leq 1}$とおいたとき対称差$D_{+}\bigtriangleup D_{-}$から境界を除いたものを描けばよい.
\subsection{2017年度}
\subsubsection*{[7]}
(1) $\abs{w}<m$ならば,
\begin{align}
    g(w)&=\frac{1}{2\pi i}\int_{\abs{\zeta}=r}\frac{\zeta f'(\zeta)}{f(\zeta)-w}\,d\zeta \\
    &=\frac{1}{2\pi i}\int_{\abs{\zeta}=r}\frac{\zeta f'(\zeta)}{f(\zeta)}\frac{1}{1-w/f(\zeta)}\,d\zeta\\
    &=\frac{1}{2\pi i}\sum_{k=0}^{\infty}\int_{\abs{\zeta}=r}\frac{\zeta f'(\zeta)}{f(\zeta)}\qty(\frac{w}{f(\zeta)})^k\,d\zeta \\
    &=\sum_{k=0}^{\infty}\qty(\frac{1}{2\pi i}\int_{\abs{\zeta}=r}\frac{\zeta f'(\zeta)}{f(\zeta)^{k+1}}\,d\zeta)w^{k}
\end{align}
が成り立つ.なお,無限和と線積分の交換に円周$\abs{\zeta}=r$における一様収束性
\begin{equation}
    \limsup_{N\to\infty}\sup_{\abs{\zeta}=r}\abs{\frac{\zeta f'(\zeta)}{f(\zeta)}\frac{1}{1-w/f(\zeta)}-\frac{\zeta f'(\zeta)}{f(\zeta)}\sum_{k=0}^N \qty(\frac{w}{f(\zeta)})^k}\leq \limsup_{N\to\infty} \frac{r}{m}\qty(\sup_{\abs{\zeta}=r}\abs{f'})\sum_{k=N+1}^\infty\qty(\frac{\abs{w}}{m})^k=0 
\end{equation}
を用いた.よって$g$は整級数で表されるので正則.

(2) $h(w)=\frac{1}{2\pi i}\int_{\abs{\zeta}=r}\frac{f'(\zeta)}{f(\zeta)-w}\,d\zeta$とおく.(1)と同様に$f$は$\abs{w}<m$で正則.一方偏角の原理により$h(w)$は$\abs{\zeta}<r$における$f(\zeta)=w$の解の個数を表す.とくに$h$は離散値を取り連続なので定数.仮定より$h(0)=1$なので$h\equiv 1$.

(3) (2)より,$\abs{w}<m$に対して$\zeta f'(\zeta)/(f(\zeta)-w)$の$\abs{\zeta}<r$における極は,1位の極$z(w)$ただ1つである.よって
\begin{equation}
    h(w)=\frac{1}{2\pi i}\int_{\abs{\zeta}=r}\frac{\zeta f'(\zeta)}{f(\zeta)-w}\,d\zeta=\Residue(z(w),\ \zeta f'(\zeta)/(f(\zeta)-w))=\lim_{\zeta\to z(w)}(\zeta-z(w))\frac{\zeta f'(\zeta)}{f(\zeta)-w}=z(w).
\end{equation}
\subsubsection*{[8]}
(1) $g_r\leq g$より,dominated convergenceの優関数として$g$が取れる.まず$r\to\infty$の極限を考える.$\abs{g}<\infty$ a.e. より$g_r(x)=\min\qty{g(x),\ r}\to g(x)$ a.e.より.次に$r\to+0$の極限は$g_r(x)\to 0 $ a.e.がわかる($g>0,\ =0$となる点をそれぞれ考えよ).

(2) 可測集合$A$および$r>0$に対して
\begin{equation}
    \int_{A}g\,d\lambda =\int_{A\cap\qty{g\leq r}}g\,d\lambda+\int_{A\cap\qty{g\geq r}}g\,d\lambda\leq r\mu(A)+\int_{\Rset}g\1_{\qty{g\geq r}}\,d\lambda 
\end{equation}
である.よって
\begin{equation}
    \limsup_{\delta\to+0}\sup_{\lambda(A)<\delta}\int_{A}g\,d\lambda\leq \int_{\Rset}g\1_{\qty{g\geq r}}\,d\lambda\quad (\forall r>0).
\end{equation}
今,$\abs{g}<\infty$ a.e.より$g(x)\1_{\qty{g\geq r}}(x)\to 0$ a.e. また$\abs{g}\1_{\qty{g\geq r}}\leq \abs{g}\in L^1$よりdominated convergenceが使えて
\begin{equation}
    \lim_{r\to\infty}\int_{\Rset}g\1_{\qty{g\geq r}}\,d\lambda=0
\end{equation}
を得る.

(3) $r>0$に対して
\begin{align}
    \int_{\Rset}\abs{f-f_n}\,d\lambda&=\qty(\int_{\qty{\abs{f-f_n}\leq r}}+\int_{\qty{\abs{f-f_n}>r}})\abs{f-f_n}\,d\lambda\\
    &\leq \int_{\Rset}\min\qty{\abs{f-f_n}, r}\,d\lambda+2\int_{\qty{\abs{f-f_n}>r}}g\,d\lambda\\
    &\leq 2\int_{\Rset}g_r\,d\lambda+2\int_{\qty{\abs{f-f_n}>r}}g\,d\lambda 
\end{align}
である.今,仮定より,任意の$\delta>0$に対して$N\in\Nset$があり$\lambda\qty(\qty{\abs{f-f_n}>r})<\delta\ (\forall n\geq N)$となる.よって$n\geq N$ならば
\begin{equation}
    \int_{\Rset}\abs{f-f_n}\,d\lambda\leq 2\int_{\Rset}g_r\,d\lambda+2\sup_{\lambda(A)<\delta}\int_{A}g\,d\lambda
\end{equation}
である.$\limsup_{n\to\infty}$としたあとに$r\to\infty,\ \delta\to+0$として結論を得る.

\subsubsection*{[9]}
(1) $x,y\in C(I),\ t\in[0,1]$に対して
\begin{equation}
    \abs{Sx(t)-Sy(t)}=\int_{0}^{t^2}\abs{A(s)}\abs{x(s)-y(s)}\,ds\leq \norm{A}\norm{x-y}
\end{equation}
より$\norm{Sx-Sy}\leq \norm{A}\norm{x-y}$を得る.

(2) 
\begin{align}
    \abs{S^mx(t)-S^my(t)}&=\abs{\int_{0}^{t^2}ds_1\,(S^{m-1}x(s_1)-S^{m-1}y(s_1))}\\
    &\leq \int_{0}^{t^2}ds_1\,\int_{0}^{s_1^2}ds_2\,\cdots\int_{0}^{s_{m-1}^2}ds_m \abs{A(s_m)}\abs{x(s_m)-y(s_m)}\\
    &\leq \norm{x-y}\norm{A}\int_{0}^{1}ds_1\,\int_{0}^{s_1}ds_2\,\cdots\int_{0}^{s_{m-1}}ds_m\\
    &=\norm{x-y}\norm{A}\frac{1}{m!}
\end{align}
より$\norm{A}/m_0!<1/2$くらい$m_0$を大きく取れば$m\geq m_0$に対して$\norm{S^mx-S^my}\leq \frac{1}{2}\norm{x-y}$となる.

(3) $x_n\in C(I)$を,
\begin{equation}
    x_0=0,\quad x_{n+1}=Sx_n 
\end{equation}
により定める.つまり$x_n=S^nx_0$.任意の$n\geq m_0$に対して
\begin{equation}
    \norm{x_{n+1}-x_n}=\frac{1}{2}\norm{x_n-x_{n-1}}\leq \cdots \leq\frac{1}{2}\norm{x_1-x_0}=\frac{\abs{\alpha}}{2^n}
\end{equation}
となる.よって$n>m\geq m_0$に対して
\begin{equation}
    \norm{x_n-x_m}\leq\sum_{k=m}^{n-1}\norm{x_{k+1}-x_k}\leq 2\abs{\alpha} (2^{-m}-2^{-n})\to0\quad (m,n\to\infty)
\end{equation}
となる.$C(I)$は完備なのでCauchy列$\qty(x_n)$はある$x\in C(I)$に収束する.
\begin{equation}
    \norm{Sx-x}\leq \norm{S(x-x_n)}+\norm{x_{n+1}-x}\leq M\norm{x-x_n}+\norm{x_{n+1}-x}\to0\quad (n\to\infty)
\end{equation}
より$Sx=x$を得る.$\xi$が他の固定点,つまり$S\xi=\xi$だとする.このとき
\begin{equation}
    \norm{x-\xi}=\norm{S^{m_0}x-S^{m_0}\xi}\leq \frac{1}{2}\norm{x-\xi}
\end{equation}
より$x=\xi$.よって固定点は一意.
\subsection{2018年度}
\subsubsection*{[6]}
(1) 
\begin{equation}
    \abs{\int_{L_{\pm}(R)}e^{-z^2/2}\,dz}=\abs{\int_{0}^{a}e^{-(R^2\pm 2Rti-t^2)/2}i\,dt}\leq \int_{0}^{a}e^{(-R^2+t^2)/2}\,dt\to0\quad (R\to\infty)
\end{equation}

(2) $D_R=\qty{z\mid \abs{\Re z}\leq R,\ 0\leq \Im z\leq a},\ I_R=[-R,R],\ J_R=\qty{s+ai\mid -R\leq s\leq R}$とおく.
\begin{equation}
    \int_{J_R}e^{-z^2/2}\,dz=\int_{-R}^{R}e^{-(s^2-a^2+2asi)/2}\,ds=e^{a^2/2}\int_{-R}^{R}e^{-s^2/2}\qty(\cos as-i\sin as)\,ds
\end{equation}
一方,
\begin{equation}
    0=\int_{\partial D_R}e^{-z^2/2}\,dz=\int_{-R}^{R}e^{-s^2/2}\,ds-e^{a^2/2}\int_{-R}^{R} e^{-s^2/2}(\cos as-i\sin as)\,ds+\qty(\int_{L_+}-\int_{L_-})e^{-z^2/2}\,dz
\end{equation}
なので,$R\to\infty$として実部を見ると
\begin{equation}
    \int_{-\infty}^{\infty}e^{-s^2/2}\cos as\,ds=\sqrt{2\pi}e^{-a^2/2}
\end{equation}
を得る.左辺の被積分関数は偶関数なので2で割って結論を得る.

\subsubsection*{[7]} 
(1) 
\begin{equation}
    \int_{X}e^{\lambda f(x)}\,d\mu(x)\geq \int_{\qty{x\mid f(x)>k}}e^{\lambda k}\,d\mu =e^{\lambda k}\mu(\qty{x\mid f(x)>k})
\end{equation}
より従う.

(2) $f(x)=x$とすれば
\begin{align}
    e^{\lambda k}\mu((k,\infty))&\leq \int_{\Rset}e^{\lambda x}\,d\mu(x)\\
    &=\int_{\Rset}e^{\lambda x}\frac{1}{\sqrt{2\pi}} e^{-x^2/2}\,dx \\
    &=\frac{1}{\sqrt{2\pi}}e^{\lambda^2/2}\int_{\Rset}e^{-(x-\lambda)^2/2}\,dx  \\
    &=e^{\lambda^2/2}
\end{align}
を得る.とくに$\lambda=k$として結論を得る.

\subsubsection*{[8]}
(1) $x(t)=e^{-2t}(x_0\cos t+(2x_0+x_1)\sin t)$.

(2) $\bm{x}=\transpose{(x,\ x')}$は
\begin{equation}
    \bm{x}'=\begin{bmatrix}
        0 & 1 \\
        -5 & -4
    \end{bmatrix}\bm{x}+\bm{g}(\bm{x}),\quad \bm{g}(x,y)=\begin{bmatrix}
        0 \\
        f(x,y)
    \end{bmatrix}
\end{equation}
である.線形項について行列の固有値の実部はすべて負なので,次の補題より原点は局所漸近安定平衡点.
\begin{lem}
    $n$次元非線形自励系$u'=Au+g(u)$を考える.ただし$A\in \mathrm{M}_n(\Rset)$であり,$g\in C(\Rset^n;\Rset^n)$は局所Lipschitz連続かつ$g(p)=o(p)\ (p\to0)$を満たすものとする.このとき
    \begin{equation}
        \Lambda\coloneq \max\qty{\Re \lambda\mid \text{$\lambda\in\Cset$は$A$の固有値}}<0
    \end{equation}
    ならば,自励系の平衡点$0$は指数安定である.
\end{lem}
\begin{proof}
    {\color{red}追記予定.}
\end{proof}
\subsection{2019年度}
\subsubsection*{[5]}
(1-1) 次の\textbf{鏡像原理}を思い出す:$D_1$を上半平面$\mathbb{H}$内の領域とし,$\partial D_1\cap \Rset$はある開区間$I$を含むとする.$f\in C(D_1\cup I)$は$D_1$内で正則で,$f(x)\in\Rset\ (x\in I)$ならば,$f$は$D_1\cup I\cup D_2$まで正則に拡張される.ここで$D_2=\qty{z\in\Cset\mid \overline{z}\in D_1}$.

問題の解答に戻る.鏡像原理より,$f$は$\qty{z\in \Cset\mid \mathrm{Re}(z)>0}$まで正則に,$\qty{z\in\Cset\mid \mathrm{Re}(z)\geq0}\setminus\qty{0}$まで連続に拡張される.さらに$g(z)=if(iz)$に対して鏡像原理を用いることで,$f$は$\Cset\setminus\qty{0}$にまで正則に拡張される.条件3より,$0$は$f$の除去可能特異点である(Riemannの除去可能定理).よって$f$は$\Cset$上の正則関数$F$に拡張される.

(1-2) したがって,$\lim_{z\to0}f(z)$は存在する.
\begin{equation}
    \lim_{z\to0}f(z)=\lim_{x\to+0}f(x)=\lim_{r\to+0}f(re^{i\pi/2})\in \Rset\cap(i\Rset)=\qty{0}
\end{equation}
より$\lim_{z\to0}f(z)=0$を得る.

(2) 単連結領域$\Lambda(\theta_0)$は$0$を含まないので,$z^{2\theta_0/\pi}=e^{\frac{2\theta_0}{\pi}\log z}$の一価な分枝が取れる.$g(z)=f(z^{2\theta_0/\pi})$は$\Lambda(\pi/2)$上定義されるので,(1)に帰着する.
\subsubsection*{[6]}
(1) (a) $\varepsilon=1>0$に対して$K>0$があり,
\begin{equation}
    \sup_{n\in\Nset}\int_{\qty{\abs{f_n}\geq K}}\abs{f_n}\,d\mu<1
\end{equation}
となる.一方,
\begin{equation}
    \sup_{n\in\Nset}\int_{\qty{\abs{f_n}\leq K}}\abs{f_n}\,d\mu\leq K\cdot\mu(X)
\end{equation}
である.よって
\begin{equation}
    \sup_{n\in\Nset}\int_{X}\abs{f_n}\,d\mu\leq 1+K\mu(X)
\end{equation}
を得る.

(b) 任意の$A\in\mathcal{B},\ K>0$に対して,
\begin{align}
    \int_{A}\abs{f_n}\,d\mu&=\int_{A\cap \qty{\abs{f_n}\geq K}}\abs{f_n}\,d\mu+\int_{A\cap\qty{\abs{f_n}\leq K}}\abs{f_n}\,d\mu\\
    &\leq \int_{\qty{\abs{f_n}\geq K}}\abs{f_n}\,d\mu+K\mu(A)
\end{align}
である.よって,
\begin{equation}
    \limsup_{\delta\to+0}\sup_{A\in\mathcal{B},\ \mu(A)<\delta}\sup_{n\in\Nset}\int_{A}\abs{f_n}\,d\mu\leq \sup_{n\in\Nset}\int_{\abs{f_n}\geq K}\abs{f_n}\,d\mu 
\end{equation}
である.$K\to\infty$として結論を得る.

(2) まず,
\begin{equation}
    M\coloneq \sup_{f\in\mathcal{C}}\int_{X}\abs{f}^p\,d\mu\geq \sup_{f\in\mathcal{C} }\int_{\qty{\abs{f}\geq K}}K^p\,d\mu=K^p\sup_{f\in \mathcal{C}}\mu(\qty{\abs{f}\geq K})
\end{equation}
である.$p'$を$p$のHölder共役指数として($p^{-1}+p'^{-1}=1$)
\begin{align}
    \int_{\qty{\abs{f}\geq K}}\abs{f}\,d\mu&=\int_{X}\abs{f}\cdot \1_{\qty{\abs{f}\geq K}}\\
    &\leq \qty(\int_{X}\abs{f}^p\,d\mu)^{1/p}\mu\qty(\qty{\abs{f}\geq K})^{1/p'}\\
    &\leq \frac{M}{K^{p-1}}
\end{align}
となる.したがって
\begin{equation}
    \limsup_{K\to\infty}\sup_{f\in \mathcal{C}}\int_{\qty{\abs{f}\geq K}}\abs{f}\,d\mu=0
\end{equation}
を得る.
\subsubsection*{[7]}
(1) $x\in H$に対して,Besselの不等式より
\begin{equation}
    \sum_{n=1}^\infty\abs{\langle x,e_n\rangle}^2\leq \norm{x}^2
\end{equation}
である.よって$\langle x,e_n\rangle\to0\ (n\to\infty)$である.

(2) 
\begin{equation}
    \norm{n^{-\alpha}\sum_{j=1}^n e_n}^2=n^{-2\alpha}\cdot n=n^{1-2\alpha}\to0\quad (n\to\infty)
\end{equation}

(3) $x\in H,\ m<n$に対して
\begin{align}
    \abs{\langle x,\ n^{-1/2}\sum_{j=1}^n e_j\rangle}&\leq n^{-1/2}\qty(\sum_{j=1}^m\abs{\langle x,e_j\rangle}+\sum_{j=m+1}^n \abs{\langle x,e_j\rangle})\\
    &\leq n^{-1/2}\qty(m\norm{x}+\qty(\sum_{j=m+1}^n \abs{\langle x,e_j\rangle}^2)^{1/2}(n-m)^{1/2})\\
    &=n^{-1/2}m\norm{x}+\qty(1-\frac{n}{n})^{1/2}\qty(\sum_{j=m+1}^n \abs{\langle x,e_j\rangle}^2)
\end{align}
よって 
\begin{equation}
    \limsup_{n\to\infty}\abs{\langle x,\ n^{-1/2}\sum_{j=1}^n e_j\rangle}\leq \qty(\sum_{j=m+1}^{\infty} \abs{\langle x,e_j\rangle}^2)
\end{equation}
である.$m\to\infty$として結論を得る.

\subsubsection*{[8]}
(1) $u$は狭義単調増加であることに注意する.

$t_0$を極大延長可能時刻,つまり$t_0=\sup\qty{t>0\mid \text{$u$は$[0,t)$で定義される}}$とする.
\begin{equation}
    \frac{u'(t)}{f(u(t))}=1
\end{equation}
より,
\begin{equation}
    \int_{0}^{u(t)}\frac{dx}{f(x)}=t 
\end{equation}
である.もし$t_0<\infty$ならば,$\lim_{t\to t_0-0}u(t)=\infty$である\footnote{もし$\lim_{t\to t_0-0}u(t)<\infty$ならば,$f$は$[0,t_0]$に拡張され,さらに局所解の存在より$[0,t_0+\varepsilon)\ (\exists\varepsilon>0)$に拡張される.これは極大性に矛盾.}.よって
\begin{equation}
    \infty=\lim_{t\to t_0-0}\int_{0}^{u(t)}\frac{dx}{f(x)}=t_0<\infty 
\end{equation}
となり矛盾.
\begin{equation}
    \limsup_{t\to\infty}\int_{0}^{u(t)}\frac{dx}{f(x)}=\infty 
\end{equation}
より$u(t)\to\infty\ (t\to\infty)$.

(2) もし$u_T\coloneq \lim_{t\to T-0}u(t)<\infty$ならば,
\begin{equation}
    \int_{0}^{u_T}\frac{dx}{f(x)}=T=\int_{0}^{\infty}\frac{dx}{f(x)}
\end{equation}
これは$f>0$に反する.

(3) $L=\limsup_{x\to\infty}\frac{f(x)}{x}<\infty$とすると,ある$M>0$について,$\frac{f(x)}{x}\leq L\ (\forall x\geq M)$となる.

もし$u\leq M\ (\forall t\geq0)$ならば,$e^{-t}u(t)\to0\ (t\to\infty)$である.

ある$t_0>0$について$u(t)\geq M\ (\forall t>t_0)$ならば,
\begin{equation}
    \qty{e^{-Lt}u(t)}'=e^{-Lt}(u'(t)-Lu(t))=e^{-Lt}(f(u(t))-Lu(t))\leq 0
\end{equation}
である.よって
\begin{equation}
    e^{-Lt}u(t)\leq e^{-Lt_0}u(t_0)\quad (t\geq t_0)
\end{equation}
したがって$e^{-2Lt}u(t)\leq e^{-L(t_0+t)}u(t_0)\to 0\ (t\to\infty)$.
\subsection{2020年度}
\subsubsection*{[5]}
(1) $I=1/(1-\lambda)$.(2) $z+z^2e^{-2019z}$など.

\subsubsection*{[6]}
(1) $\qty{a_n}_n\in l^1$とする.このとき$a_n\to0$であるから$\qty{n\mid \abs{a_n}\geq 1}$は有限集合である.よって
\begin{equation}
    \sum_{n=1}^\infty \abs{a_n}^p=\sum_{n\colon \abs{a_n}\geq 1}\abs{a_n}^p+\sum_{n\colon \abs{a_n}\leq 1}\abs{a_n}^p \leq \sum_{n\colon \abs{a_n}\geq 1}\abs{a_n}^p+\sum_{n\colon \abs{a_n}\leq 1}\abs{a_n}<\infty 
\end{equation}
となる.

(2) $\alpha>1/2$のとき,$\qty{a_n}_n\in l^2$に対してHölderの不等式より 
\begin{equation}
    \norm{T_\alpha \qty{a_n}_n}_{l^1}=\sum_{n\geq1} n^{-\alpha}\abs{a_n}\leq \qty(\sum_{n\geq 1}n^{-2\alpha})^{1/2}\norm{\qty{a_n}_n}_{l^2}
\end{equation}
だからである.

(3) 答えはNo.$a_n=1/(n^{1/2}\log n)\ (n\geq 2)$とおくと$\qty{a_n}_n\in l^2$だが,$T_{1/2}\qty{a_n}_n\notin l^1$である.

(4) $N=1,2,\ldots$に対して$T^{(N)}_{\alpha}\colon l^2\to l^1$を
\begin{equation}
    \qty(T^{(N)}_{\alpha}\qty{a_n}_n)_n=\begin{cases}
        n^{-\alpha}a_n & (n=1,\ldots,N) \\
        0 & (n>N)
    \end{cases}
\end{equation}
と定める.$T^{(N)}_{\alpha}$はfinite-rank operator (像が有限次元)であるからコンパクト作用素.一方,
\begin{equation}
    \norm{(T_{\alpha}-T^{(N)}_{\alpha})\qty{a_n}_n}_{l^1}=\sum_{n\geq N+1}n^{-\alpha}\abs{a_n}\leq \qty(\sum_{n\geq N+1}n^{-2\alpha})^{1/2}\norm{\qty{a_n}_{n}}_{l^2}
\end{equation}
であるから,
\begin{align}
    \norm{T_{\alpha}-T^{(N)}_{\alpha}}_{\mathcal{L}(l^2,l^1)}
    &=\sup_{\qty{a_n}_n\in l^2,\ \norm{\qty{a_n}_{n}}_{l^2}
    \leq 1}\norm{(T_{\alpha}-T^{(N)}_{\alpha})\qty{a_n}_n}_{l^1}\\
    &\leq \qty(\sum_{n\geq N+1}n^{-2\alpha})^{1/2}\to0\quad (N\to\infty)
\end{align}
である.よって,$T_\alpha$は,コンパクト作用素$T_{\alpha}^{(N)}$の作用素ノルム$\norm{\bullet}_{\mathcal{L}(l^2,l^1)}$に関する極限としてコンパクト作用素である.
\subsubsection*{[7]}
(1) 正しくない.$I_n=[-1/n,1/n]$として$f_n(x)=\1_{\Rset\setminus I_n}/x^3$とおくと$f_n\in L^1(\Rset)$は$f(x)=1/x^3$にほとんど至る所収束する.例えば$\varphi(x)=\begin{cases}
    1-\abs{x} & (\abs{x}\leq 1) \\
    0 & (\abs{x}>1)
\end{cases}$とおくと,
\begin{equation}
    \int_{\Rset}f_n\varphi=0\quad (n\geq 1)
\end{equation}
である.一方$f\varphi$は可積分ではない.

(2) 正しい.$A=\qty{x\in\Rset\mid f_n(x)\nrightarrow f(x)}$として,
$n\geq1,\ \varepsilon>0$に対して
\begin{equation}
    A_{n,\varepsilon}=\qty{x\in\Rset\mid \abs{f(x)-f_n(x)}\geq \varepsilon}
\end{equation}
とおく.
\begin{equation}
    A=\bigcup_{\varepsilon\in \Qset_{>0}}\bigcap_{n\in\Nset}\bigcup_{m\geq n}A_{m,\varepsilon}
\end{equation}
である.今$m(A)=0$なので,測度の単調収束定理より任意の$\varepsilon>0$に対して
\begin{equation}
    0=m(A\cap[-1/\varepsilon,1/\varepsilon])\geq m\qty(\bigcap_{n\in\Nset}\bigcup_{m\geq n}A_{m,\varepsilon}\cap[-1/\varepsilon,1/\varepsilon])=\lim_{n\to\infty}m\qty(\bigcup_{m\geq n}A_{m,\varepsilon}\cap[-1/\varepsilon,1/\varepsilon])
\end{equation}
である.

(3) 正しい.$\Phi$は連続なので$\Phi\circ f_n$は$\Phi\circ f$にほとんど至るところ収束する.一方$\int_{\Rset}\Phi(f_n(x))\,dx\to0$より,ある部分列$\Phi\circ f_{n_k}$は0にほとんど至るところ収束する.よって$\Phi\circ f=0$ a.e.を得る.

(4) 正しい.$\Psi(x,f_n(x))$は$\Psi(x,f(x))$にa.e.で収束する.
\begin{equation}
    K\coloneq \mathop{\mathrm{pr}}(\supp\Psi)=\qty{x\in\Rset\mid \exists y\in\Rset \text{ s.t. }(x,y)\in \supp\Psi}
\end{equation}
とおくと
\begin{equation}
    \abs{\Psi(x,f_n(x))}\leq M\1_{K}
\end{equation}
である.ここで$M=\sup_{\Rset^2}\abs{\Psi}$.今$K$はコンパクトなので$M\1_{K}$は可積分.よってdominated convergenceにより結論が得られる.

\subsubsection*{[8]}
(1) $f(t)=\int_{0}^{t}u(s)\,ds$とおくと,$u$は$C^1$級で$u(t)\leq f(t)+c$.
\begin{equation}
    \qty{e^{-t}(f(t)+c)}'=e^{-t}\qty(u(t)-f(t)-c)\leq 0
\end{equation}
となるので$e^{-t}(f(t)+c)$は単調非増加.$t=0$での値を考えて$e^{-t}(f(t)+c)\leq c$を得る.よって
\begin{equation}
    u(t)\leq f(t)+c\leq ce^t 
\end{equation}

(2) $f(t)=\int_{0}^t(s+1)^{-1}u(s)\,ds$とおくと$u(t)\leq f(t)+(t+1)^2$である.
\begin{equation}
    \qty{\frac{f(t)}{t+1}}'=\frac{f'(t)(t+1)-f(t)}{(t+1)^2}=\frac{u(t)-f(t)}{(t+1)}^2\leq 1 
\end{equation}
より$(t+1)^{-1}f(t)\leq t$を得る.よって
\begin{equation}
    u(t)\leq f(t)+(t+1)^2\leq t(t+1)+(t+1)^2=2t^2+3t+1
\end{equation}
である.

(3) $f(t)=\int_{0}^tu(s)^2\,ds$とおき,$u(t)\geq f(t)+1$となる$u$があったとする.このとき$0<t<1$に対して$g(t)=(1-t)f(t)-t$とおくと,
\begin{equation}
    g'(t)=-f(t)+(1-t)f'(t)-1\geq -f(t)+(1-t)(1+f(t))^2-1=(1+f(t))g(t)\geq g(t)
\end{equation}
であり,よって
\begin{equation}
    \qty{e^{-t}g(t)}'=e^{-t}(g'-g)\geq0 
\end{equation}
を得る.$g(0)=f(0)=0$より$g(t)\geq 0$を得る.これは
\begin{equation}
    f(t)\geq \frac{t}{1-t}\quad (0<t<1)
\end{equation}
であることを示している.よって
\begin{equation}
    u(t)\geq \frac{t}{1-t}+1
\end{equation}
であり,$t\to1-0$の極限で$u(t)\to+\infty$となる.これは矛盾.
\subsection{2021年度}
\subsubsection*{[5]}
(ii)ならば(i)は,留数定理による.(i)ならば(ii)を示す.まず次を示す:($\diamondsuit$)「$g\in C(\overline{D})$が有理型で境界には極を持たず,任意の多項式$P$に対して
\begin{equation}
    \int_{C}P(z)g(z)\,dz=0 
\end{equation}
ならば,$g$は正則である」.$z_1,\ldots,z_k$が$g$の$m_1,\ldots,m_k$位の極であるとする.このとき
\begin{equation}
    P(z)=(z-z_1)^{m_1-1}(z-z_2)^{m_2}\cdots (z-z_k)^{m_k}=\frac{\prod_{j=1}^k(z-z_j)^{m_j}}{z-z_1}
\end{equation}
とおくと,留数定理より
\begin{equation}
    \int_{C}P(z)g(z)\,dz=\Residue(z_1,P(z)f(z))\neq0 
\end{equation}
である.これは矛盾.

さて,$f$が(i)を満たすとする.このとき$g(z)=f(z)-1/(2\pi iz)$は有理型で,($\diamondsuit$)の仮定を満たす.よって$g$は正則であり,$f=g+1/(2\pi iz)$は(ii)を満たす.

\subsubsection*{[6]}\label{pm_2021_6}
(1) 
\begin{equation}
    \int_{\Rset}(u_n(x)-a_1)^2(u_n(x)-a_2)^2\,dx\geq \int_{A_{\varepsilon,n}}\varepsilon^2\cdot \varepsilon^2\,dx=\varepsilon^4m(A_{\varepsilon,n})
\end{equation}
より
\begin{equation}
    m(A_{\varepsilon,n})\leq \frac{1}{\varepsilon^4}\int_{\Rset}(u_n(x)-a_1)^2(u_n(x)-a_2)^2\,dx\to0\quad (n\to\infty)
\end{equation}
を得る.

(2) 
\begin{equation}
    A=\qty{x\in \Rset\,\middle|\, \liminf_{n\to\infty}\abs{u_n(x)-a_j}>0\ (j=1,2)}
\end{equation}
とおく.任意の$l\in\Zset$に対して$m(A\cap [l,l+1))=0$であることをいえばよい.
\begin{equation}
    A\cap[l,l+1)=\bigcup_{k\geq 1}\bigcup_{N\geq 1}\bigcap_{n\geq N}(A_{1/k,n}\cap [l,l+1))
\end{equation}
である.任意の$k,N\geq 1$に対して,測度の単調収束定理より
\begin{equation}
    m\left(\bigcap_{n\geq N}A_{1/k,n}\cap[l,l+1) \right)=\lim_{n\to\infty}m(A_{1/k,n}\cap[l,l+1))=0 
\end{equation}
なので結論を得る.

(3) 偽らしい(未検証).

\subsubsection*{[7]}
(1) $u\in X$に対して
\begin{equation}
    |Au(x)|\leq \int_{0}^{x}\abs{u(y)}\,dy\leq \norm{u}_{X}\quad (x\in [0,1])
\end{equation}
よって$\norm{Au}_{X}\leq \norm{u}_{X}$を得る.

(2) 
\begin{equation}
    \abs{A^nu(x)}\leq \int_{0}^{x}dy_1\int_{0}^{y_1}dy_2\cdots\int_{0}^{y_{n-1}}dy_{n}\abs{u(y_{n})}\leq \frac{1}{n!}\norm{u}_{X}
\end{equation}
である.

(3) 
\begin{equation}
    T_N=z\sum_{n=0}^N\qty(\frac{A}{z})^n 
\end{equation}
とおく.これは$\mathcal{L}(X)$のCauchy列である:実際,$N>M$に対して
\begin{equation}
    \norm{T_N-T_{M}}\leq \abs{z}\sum_{n=M+1}^N\frac{\norm{A^n}}{\abs{z}^n}\leq \abs{z}\sum_{n=M+1}^{N}\frac{1}{n!\abs{z}^n}\to0\quad (M,N\to\infty)
\end{equation}
よって$T_N$はある有界線形作用素$T$に作用素ノルムで収束する.$T$が求める逆作用素であることが確かめられる.
\subsection{2022年度}
\subsubsection*{[6]}
(1) $k\in\Nset$に対して
\begin{equation}
    G_k\coloneq \qty{x\in X\mid \abs{g(x)}\geq k}
\end{equation}
とおく.
\begin{equation}
    \norm{g}_2^2\geq \int_{G_k}\abs{g}^2\geq k^2\mu(G_k)
\end{equation}
より$\mu(G_k)\leq \norm{g}_2^2/k^2$となる.$A\in \mathcal{F}$に対して
\begin{equation}
    \int_{A}\abs{g}^2=\int_{A\setminus G_k}\abs{g}^2+\int_{A\cap G_k}\abs{g}^2\leq k^2\mu(A)+\int_{X}\abs{g}^2\1_{G_k}
\end{equation}
なので,
\begin{equation}
    \limsup_{\delta\to+0}\sup_{A\in \mathcal{F},\ \mu(A)<\delta}\int_{A}\abs{g}^2\leq \int_{X}\abs{g}^2\1_{G_k}
\end{equation}
である.今,$\abs{g}<\infty$ a.e.より$g(x)\1_{G_k}\to 0$ a.e.である.また$\abs{g}^2\1_{G_k}\leq \abs{g}^2\in L^1(X)$よりdominated convergenceが使えて
\begin{equation}
    \lim_{k\to\infty}\int_{X}\abs{g}^2\1_{G_k}=0 
\end{equation}
となる.これで示された.

(2) $f_n$が零集合$A$を除いて$f$に一様収束しているとは,「任意の$k\geq 1$に対して$n_k\geq 1$があり$E_{n_k,k}\subset A$となる」ことである.

(3) $f_n\to f\ $a.e. in $X$とは,
\begin{equation}
    \mu\qty(\limsup_{n\to\infty}E_{n,k})=0\quad (\forall k\geq 1)
\end{equation}
ということである.測度の単調収束定理より
\begin{equation}
    \lim_{N\to\infty}\mu\qty(\bigcap_{n\geq N}E_{n,k})=0.
\end{equation}
このことから求める$n_k$は作れる.

(4) $g\in L^2(X)$とする.$\qty{f_n}_n$の任意の部分列$\qty{f_{n_k}}_k$に対してさらなる部分列$\qty{f_{n_{k_l}}}_l$が存在して,
\begin{equation}
    \lim_{l\to\infty}\int_{X}f_{n_{k_l}}g=\int_{X}fg 
\end{equation}
となることを示す.

$\varepsilon>0$とする.(1)より,ある$\delta>0$が存在して,
\begin{equation}
    A\in \mathcal{F},\ \mu(A)<\delta\Longrightarrow \int_{A}\abs{g}^2\,d\mu<\varepsilon
\end{equation}
とできる.$f_{n_k}\to f$ a.e.より,ある部分列$\qty{k_l}_l$を取れば,
\begin{equation}
    \mu\qty(\bigcup_{l\geq1}E_{n_{k_l},l})<\delta 
\end{equation}
とできる.よって
\begin{align}
    \abs{\int_{X}(f-f_{n_{k_l}})g}&\leq \int_{E_{n_{k_l},l}}\abs{f-f_{n_{k_l}}}\abs{g}+\int_{X\setminus E_{n_{k_{l}},l}}\abs{f-f_{n_{k_l}}}\abs{g}\\
    &\leq 2M^{1/2}\norm{g}_{L^2(E_{n_{k_l}})}+\norm{f-f_{n_{k_{l}}}}_{L^2(X\setminus E_{n_{k_{l}}})}\norm{g}_{L^2(X)}\\
    &\leq 2M^{1/2}\varepsilon^{1/2}+\frac{1}{l}\mu(X)^{1/2}\norm{g}_{L^2(X)}
\end{align}
となる.$\limsup_{l\to\infty}$を取った後$\varepsilon\to0$とせよ.

\subsubsection*{[7]}
(1) $f\in L^r$に対して,Hölderの不等式より
\begin{equation}
    \abs{Tf(x)}=\abs{\int_{\Omega}K(x,y)f(y)\,dy}\leq \qty(\int_{\Omega}\abs{K(x,y)}^p\,dy)^{1/p}\qty(\int_{\Omega}\abs{f(y)}\,dy)^{1/r}
\end{equation}
である.なお,$K\in L^p(\Omega\times \Omega)$よりほとんどすべての$x\in \Omega$に対して$K(x,-)\in L^p(\Omega)$である(Fubiniの定理).よって
\begin{align}
    \int_{\Omega}\abs{Tf(x)}^p\,dx&=\int_{\Omega}\,dx \qty(\int_{\Omega}\abs{K(x,y)}^p\,dy)\qty(\int_{\Omega}\abs{f(y)}\,dy)^{p/r}\\
    &=\qty(\int_{\Omega\times \Omega}\abs{K(x,y)}\,dxdy)\qty(\int_{\Omega}\abs{f(y)}\,dy)^{p/r}
\end{align}
したがって$\norm{Tf}_{p}\leq \norm{K}_p\norm{f}_r$.

(2) ほとんどすべての$x\in \Omega$に対して$K(x,-)\in L^p(\Omega)$であるから,そのような$x$に対して
\begin{equation}
    T_nf(x)=\int_{\Omega}K(x,y)f_n(y)\,dy \to \int_{\Omega}K(x,y)f(y)\,dy=Tf(x)
\end{equation}

(3) $L^r$はreflexiveであるから「$f_n\rightharpoonup f$ weakly in $L^r$ならば$Tf_n\to Tf$ strongly in $L^p$」を示せばよい.弱収束列は有界なので$M=\sup_{n}\norm{f_n}_r<\infty$.また(2)より$Tf_n\to Tf$ a.e. またa.e. $x\in \Omega$に対して
\begin{equation}
    \abs{Tf_n(x)}\leq \int_{\Omega}\abs{K(x,y)}\abs{f_n(y)}\,dy\leq M\norm{K(x,-)}_{p}
\end{equation}
であり,$K\in L^p(\Omega\times \Omega)$より右辺は($x$に関して)$L^p$に属する.よってdominated convergenceより$Tf_n\to Tf$ strongly in $L^p$が従う.
\subsubsection*{[8]}
$f(z)=1/(z^2\sin z)$の正方形領域$\qty{x+iy\mid \max\qty{\abs{x},\abs{y}}\leq (N+1/2)\pi}$における極は,$z=n\pi\ (n\in \Zset\cap\qty[-n,n])$である.極の位数は$0$でのみ$2$で,$n\pi\ (n\neq0)$では$1$.よって
\begin{equation}
    \Residue(0,f)=\frac{1}{2}\lim_{z\to0}\frac{d^2}{dz^2}z^3f(z)=\frac{1}{2}\lim_{z\to0}\frac{d^2}{dz^2}\frac{z}{\sin z}=\frac{1}{6},
\end{equation}
\begin{equation}
    \Res(n\pi,f)=\lim_{z\to n\pi}(z-n\pi)f(z)=\lim_{z\to n\pi}\frac{z-n\pi}{z^2\sin z}=\frac{(-1)^n}{(n\pi)^2}\quad (n\neq0)
\end{equation}
であるから,留数定理より
\begin{equation}
    \frac{1}{2\pi i}\int_{C_N}\frac{dz}{z^2\sin z}=\frac{1}{3}+\frac{2}{\pi^2}\sum_{n=1}^N\frac{(-1)^n}{n^2}
\end{equation}
を得る.

(2) 答えは$-\pi/12$.$C_N$上$\abs{\sin z}\geq 1$がわかる.よって
\begin{equation}
    \abs{\frac{1}{2\pi i}\int_{C_N}\frac{dz}{z^2\sin z}}\leq \frac{4(2N+1)\pi}{2\pi i}\cdot \frac{1}{\qty(\qty(N+1/2)\pi)^2}\to0\quad (N\to\infty)
\end{equation}
であるから,(1)の結果と合わせて結論を得る.
\subsection{2023年度}
\subsubsection*{[5]}
(1) 任意の$\rho>0$に対して,$N=N_{\rho}$を,$(N!)^{1/N}>2\rho/R$となるくらい大きく取れば,$\abs{z}<\rho$に対して
\begin{equation}
    \sum_{n=0}^\infty \abs{\frac{a_n}{n!}z^n}\leq \sum_{n< N}\frac{\abs{a_n}}{n!}\rho^n+\sum_{n\geq N}\abs{a_n}\qty(\frac{R}{2})^n<\infty 
\end{equation}
これは任意の$\rho>0$に対して$F(z)$が$\abs{z}<\rho$で絶対収束することを示している.

(2), (3) Cauchyの積分公式より
\begin{equation}
    a_n=\frac{1}{2\pi i}\int_{\abs{u}=r}\frac{f(u)}{u^{n+1}}\,du 
\end{equation}
が成り立っている.
\begin{align}
    F(z)&=\sum_{n\geq0}\frac{a_n}{n!}z^n\\
    &=\sum_{n\geq0}\frac{z^n}{2\pi i n!}\int_{\abs{u}=r}\frac{f(u)}{u^{n+1}}\,du\\
    &=\frac{1}{2\pi i}\int_{\abs{u}=r}\frac{f(u)}{u}\sum_{n\geq0}\frac{1}{n!}\qty(\frac{z}{\zeta})^n\\
    &=\frac{1}{2\pi i}\int_{\abs{u}=r}\frac{f(u)}{u}e^{z/u}\,du 
\end{align}
\subsubsection*{[6]}
(1) いいえ.$f_n(x)=x^{-1/p}/n$とおくと
\begin{equation}
    \int_{I}\abs{f_n(x)}\,dx=\frac{1}{n}\frac{1}{1-p^{-1}}\to 0\quad (n\to0) 
\end{equation}
だが,$f_n\notin L^p$である.

(2) 正しい.部分列に移って
\begin{equation}
    \norm{f_n-f_{n+1}}_{L^1(I)}\leq 2^{-n}
\end{equation}
だとしてよい.
\begin{equation}
    g_n(x)\coloneq\sum_{k=1}^{n-1}\abs{f_k(x)-f_{k+1}(x)}
\end{equation}
とおくと,
\begin{equation}
    \int_{I}g_n\,dx\leq \sum_{k=1}^{n-1}2^{k-1}\leq 1 
\end{equation}
$g_n$は単調増加なのである可測関数$g$に収束するが,Fatouの補題より
\begin{equation}
    \int_{I}g\,dx\leq \liminf\int_{I}g_n\,dx\leq 1 
\end{equation}
なので$g\in L^1(I)$である.とくに$g<\infty\ \text{a.e. in $I$}$.したがって
\begin{equation}
    \abs{f_n(x)-f_m(x)}\leq \sum_{k=m}^{n-1}\abs{f_{k+1}(x)-f_k(x)}=g_n(x)-g_m(x)\to 0 \quad \text{a.e. in $I$}\quad(m,n\to\infty)
\end{equation}
なので$f$はa.e.でCauchy列であるからある$f^\ast$に収束する.$\norm{f-f_n}_{L^1}\to0$であったから$f=f^\ast$ a.e.である.

(3) 正しい.任意の部分列$\qty{n_k}_k$について,$\norm{f-f_{n_k}}_{L^1}\to0\ (k\to\infty)$だから,更なる部分列$\qty{k_l}_l$があり$f_{n_{k_l}}\to f$ a.e.となる.$g$は連続より$g(f_{n_{k_l}}(x))\to g(f(x))$ a.e.である.さらに$\abs{g\circ f_{n_{k_l}}}\leq \sup\abs{g}<\infty$なので,$L^p$-variant of dominated convergenceより($I$は測度有限に注意せよ)
\begin{equation}
    \norm{g(f(x))-g(f_{n_{k_{l}}}(x))}_{L^p}\to0\ (l\to\infty)    
\end{equation}
を得る.
\subsubsection*{[7]}
$M=\sup\abs{f}$とおく.

(1) $F'(t)=-e^{-\lambda t}f(t)$であることを示す.
\begin{align}
    \abs{\frac{F(t+h)-F(t)}{h}+e^{\lambda t}f(t)}
    &=\abs{-\frac{1}{h}\int_{t}^{t+h}(e^{-\lambda s}f(s)-e^{-\lambda t}f(t))\,ds}\\
    &\leq \frac{1}{\abs{h}}\abs{\int_{t}^{t+h}\qty(\abs{f(s)}(e^{-\lambda t}-e^{-\lambda s})+e^{-\lambda t}\abs{f(s)-f(t)})\,ds}\\
    &\leq M\sup_{s\in[t,t+h]}\abs{e^{-\lambda t}-e^{-\lambda s}}+\sup_{s\in[t,t+h]}\abs{f(s)-f(t)}\\
    &\to0\quad (h\to0)
\end{align}
より従う.

(2) $G(t)=\int_0^t e^{\lambda s}f(s)\,ds$とおくと$G'(t)=e^tf(t)$である.
\begin{align}
    \abs{F(t)}&\leq M\int_{t}^\infty e^{-\lambda s}\,ds=\frac{M}{\lambda}e^{-\lambda t},\\
    \abs{G(t)}&\leq M\int_{0}^{t}e^{\lambda s}\leq \frac{M}{\lambda}e^{\lambda t} 
\end{align}
に注意する.
\begin{equation}
    y(t)=-\frac{1}{2\lambda}\qty(e^{\lambda t}F(t)+e^{-\lambda t}G(t))
\end{equation}
より
\begin{equation}
    \abs{y(t)}\leq \frac{1}{2\lambda}\qty(e^{\lambda t}\frac{M}{\lambda}e^{-\lambda t}+e^{-\lambda t}\frac{M}{\lambda}e^{\lambda t})=\frac{M}{\lambda^2} <\infty 
\end{equation}
なので$y$は有界.以下$y$が(A)を満たすことを確認する:
\begin{align}
    y'&=-\frac{1}{2\lambda}\qty(e^{\lambda t}(\lambda F+F')+e^{-\lambda t}(-\lambda G+G'))\\
    y''&=-\frac{1}{2\lambda}\qty(e^{\lambda t}\qty(\lambda^2F+2\lambda F'+F'')+e^{-\lambda t}\qty(\lambda^2G-2\lambda G'+G''))\\
    &=-\frac{1}{2\lambda}\qty(\lambda^2\qty(e^{\lambda t}F+e^{-\lambda t}G)+2\lambda\qty(e^{\lambda t}F'-e^{-\lambda t}G')+\qty(e^{\lambda t}F''+e^{-\lambda t}G''))
\end{align}
ここで
\begin{equation}
    -\frac{1}{2\lambda}\qty(e^{\lambda t}F+e^{-\lambda t}G)=y,
\end{equation}
\begin{equation}
    F'=-e^{-\lambda t}f(t),\quad G'=e^{\lambda t}f(t),
\end{equation}
\begin{equation}
    F''=e^{-\lambda t}(\lambda f-f'),\quad G''=e^{\lambda t}(\lambda f+f')
\end{equation}
より
\begin{equation}
    y''=\lambda^2 y+f 
\end{equation}
を得る.

(3) (A)の解は,$y$と斉次方程式$x''-\lambda^2x=0$の一般解の和で表される:
\begin{equation}
    x(t)=Ae^{\lambda t}+Be^{-\lambda t}+y(t)
\end{equation}
$x(0)=0$より,
\begin{equation}
    A+B+\int_{0}^\infty e^{-\lambda s}f(s)\,ds=0 
\end{equation}
である.また,$x$が$[0,\infty)$で有界であるには
\begin{equation}
    A=0
\end{equation}
でなければならない.よって
\begin{equation}
    x(t)=\int_{0}^\infty \qty(e^{-\lambda\abs{s-t}}-e^{-\lambda(s+t)})f(s)\,ds
\end{equation}
が求める解.

\subsubsection*{[8]}
(1) $K^\ast$が閉であること:$x\in H$に対して$\alpha_x\colon H\to \Rset,\ \alpha_x(y)=(x,y)$は連続であるから,
\begin{equation}
    K^\ast=\bigcap_{x\in K} \qty{y\in H\mid (x,y)\leq 0}=\bigcap_{x\in K} \alpha_{x}^{-1}((-\infty,0])
\end{equation}
は閉である.$\alpha,\beta\geq0,\ y,z\in K^\ast$に対して,
\begin{equation}
    (x,\alpha y+\beta z)=\alpha(x,y)+\beta(x,z)\leq 0\quad (x\in K)
\end{equation}
となるから$\alpha y+\beta z\in K^\ast$.

(2) straight forward.

(3) $s=\inf_{\xi\in K}\norm{x-z}$とおく.任意の$k\in\Nset$に対して$\xi_k\in K$があり,
\begin{equation}
    s^2\leq \norm{\xi_k-z}<s^2+\frac{1}{n}
\end{equation}
となる.$\qty{\xi_k}_k$がCauchy列であることを示そう:$k,l\in\Nset$に対して
\begin{align}
    \norm{\xi_k-\xi_l}^2
    &=2\norm{\xi_k-z}^2+2\norm{\xi_l-z}^2-4\norm{\frac{\xi_k+\xi_l}{2}-z}^2\\
    &\leq 2\qty(s^2+\frac{1}{k})+2\qty(s^2+\frac{1}{l})-4s^2\\
    &\leq \frac{2}{k}+\frac{2}{l}\\
    &\to0\quad (k,l\to\infty)
\end{align}
である($K$は閉凸錐なので$(\xi_k+\xi_l)/2\in K$である).よって$\qty{\xi_k}_k$はある$x\in K$に収束する.$\norm{x-z}=s$である.

(4) $\xi\in K,\ \alpha,\beta\geq0$に対して$\alpha x+\beta\xi\in K$であるから
\begin{equation}
    \norm{x-z}^2\leq \norm{\alpha x+\beta \xi-z}^2 
\end{equation}
が成り立つ.$\alpha=1$としてみると,
\begin{equation}
    0\leq \norm{x+\beta\xi-z}^2-\norm{x-z}^2=\beta(\beta\norm{\xi}^2-2(y,\xi))
\end{equation}
を得る.$\beta\to +0$とすることで$(y,\xi)\leq 0$を得る.よって$y\in K^\ast$である.

一方で,$\beta=0$とすると,
\begin{equation}
    0\leq \norm{\alpha x-z}^2-\norm{x-z}^2=\qty(\alpha-1)\qty((\alpha+1)\norm{x}^2-2(x,z))
\end{equation}
である.$\alpha\to1\pm 0$とすることで$(x,z)=\norm{x}^2$つまり$(x,y)=0$を得る.
\subsection{2024年度}
\subsubsection*{[5]}\label{pm_2024_5}
(1) $Z$を$f$の$c$内部の零点全体の集合とする.積分路の変形により
\begin{equation}
    \frac{1}{2\pi i}\int_{c}\frac{f'}{f}\,dz=\frac{1}{2\pi i}\sum_{z_0\in Z}\int_{\partial B_{\rho}(z_0)}\frac{f'}{f}\,dz 
\end{equation}
である.ただし$\rho$は十分小さな正の数.$z_0\in Z$に対して,$f$の零点$z_0$の位数を$m$とする:$f(z)=(z-z_0)\varphi(z)$であり,$\varphi$は$D$で正則,$\varphi(z_0)\neq0$と書ける.
\begin{equation}
    f'(z)=m(z-z_0)^{m-1}\varphi(z)+(z-z_0)^m\varphi'(z)
\end{equation}
であるから
\begin{equation}
    \frac{f'(z)}{f(z)}=\frac{m}{z-z_0}+\frac{\varphi'(z)}{\varphi(z)}
\end{equation}
である.いま$\varphi(z_0)\neq0$より$\varphi'/\varphi$は$B_{\rho}(z_0)$で正則だから,
\begin{equation}
    \frac{1}{2\pi i}\int_{\partial B_{\rho}(z_0)}\frac{f'}{f}\,dz=m 
\end{equation}
である.

(2) 開球$B_R=B_{R}(z_0)\subset\subset D$に対して,$f$が$D$内で正則であることを示す.$c$を$B_R$内の閉曲線に対して
\begin{equation}
    \int_{c}f_n\,dz=0 
\end{equation}
が成り立つ(Cauchyの定理).$f_n$は$c$上$f$に一様収束するので
\begin{equation}
    \int_{c}f\,dz=0 
\end{equation}
となる.Moreraの定理より$f$が正則であることが従う.Cauchyの積分の公式より
\begin{equation}
    f_n'(z)=\frac{1}{2\pi i}\int_{\partial B_R}\frac{f_n(\zeta)}{(\zeta-z)^2}\,d\zeta,\quad f'(z)=\frac{1}{2\pi i}\int_{\partial B_R}\frac{f(\zeta)}{(\zeta-z)^2}\,d\zeta,\quad (z\in B_R)
\end{equation}
である.よって任意の$0<\rho<R$に対して
\begin{equation}
    \sup_{z\in B_\rho}\abs{f'(z)-f_n'(z)}\leq \abs{\frac{1}{2\pi i}\int_{B_R}\frac{f(\zeta)-f_n(\zeta)}{(\zeta-z)^2}\,dz}\leq \frac{2\pi R}{2\pi(R-\rho)^2}\sup_{\zeta\in B_R}\abs{f(\zeta)-f_n(\zeta)}\to 0 \quad (n\to\infty)
\end{equation}
を得る.よって$f_n'$は$f$に$D$内で広義一様収束する.

(3) 背理法で示す.$f$が定数ではなく,かつ単射でもないとする:$f(z_1)=f(z_2),\ z_1\neq z_2\in D$.このとき$g_n(z)=f_n(z)-f_n(z_1)$とおくと$g_n$は$g(z)=f(z)-f(z_1)$に$D$上広義一様収束する.また$g_n$は$\widetilde{D}=D\setminus\qty{z_1}$上nonzeroである.よって$B=B_\rho(z_2)\subset\subset \widetilde{D}$に対して
\begin{equation}
    \frac{1}{2\pi i}\int_{\partial B}\frac{g_n'}{g_n }\,dz =0 \quad (n\geq 1)
\end{equation}
だが,$g(z_2)=0$より
\begin{equation}
    \frac{1}{2\pi i}\int_{\partial B}\frac{g'}{g}\,dz\geq 1 
\end{equation}
である.$g_n'/g_n$は$\partial B$上$g'/g$に一様収束するのでこれは矛盾である.

\subsubsection*{[6]}
(1) 正しい.$\norm{(f-f_n)1_{\Rset\setminus E_n}}_1\to0$より,ある部分列$f_{n_k}$が存在して
\begin{equation}
    (f-f_{n_k})1_{\Rset\setminus E_{n_k}}\to 0\quad (\text{a.e. in $\Rset$})
\end{equation}
となる.
\begin{equation}
    E\coloneq \liminf_{k\to \infty} E_{n_k} =\bigcup_{k\geq1}\bigcap_{l\geq k}E_{n_l}
\end{equation}
とおくと$\mu(E)=0$である.$x\notin E$ならば,ある$k\geq 1$があり,任意の$l\geq k$に対して$x\notin E_{n_l}$となる.よって
\begin{equation}
    \abs{f(x)-f_{n_k}(x)}\leq \abs{f(x)}|1-1_{\Rset\setminus E_{n_k}}|+\abs{f(x)-f_{n_k}(x)}1_{\Rset\setminus E_{n_k}}\to 0\quad (k\to\infty )
\end{equation}
となる.

(2) 正しい.Hölderの不等式より
\begin{equation}
    \norm{f-f_n}_{1}\leq \norm{(f-f_n)1_{\Rset\setminus E_n}}_1+\norm{f_n(1_{\Rset\setminus E_n}-1)}_{1}\leq \norm{(f-f_n)1_{\Rset\setminus E_n}}_1+\norm{f_n}_2 \mu(E_n)^{1/2}\to0 
\end{equation} 

\subsubsection*{[7]}
$f\in C(D),\ x\in D$に対して
\begin{align}
    \abs{T_nf_k(x)}&\leq \int_{D}\abs{x-y}^{-1}\Psi_n(x-y)\abs{f(y)}\,dy\\
    &\leq \norm{f}\int_{\qty{y\in D\mid \abs{x-y}\geq 1/n}}n\cdot 1\,dy\\
    &\leq \pi n \norm{f} 
\end{align}
よって$\norm{T_nf}\leq \pi n\norm{f}$を得る.よって$T_n$は$C(D)$から$C(D)$への有界線形作用素であり,とくに$C(D)$の(一様)有界列$\qty{f_k}_k,\ M=\sup_{k}\norm{f_k}<\infty$に対して$\qty{T_nf_k}_k$は一様有界である.さらに$x,z\in D$に対して
\begin{align}
    \abs{T_nf_k(x)-T_nf_k(z)}&\leq \int_{D}\abs{\Phi_n(x,y)-\Phi_n(z,y)}\abs{f_k(y)}\,dy\\
    &\leq 2\pi M\sup_{y\in D}\abs{\Phi_n(x,y)-\Phi_n(z,y)}
\end{align}
だが,$\Phi_n\colon D\times D\to \Rset$はコンパクト集合上連続より一様連続であることに注意すると,
\begin{equation}
    \limsup_{\delta\to +0}\sup_{x,z\in D,\ \abs{x-z}<\delta}\abs{T_nf_k(x)-T_nf_k(z)}\leq 2\pi M\limsup_{\delta\to+0}\sup_{x,y,z\in D,\ \abs{x-z}<\delta}\abs{\Phi_n(x,y)-\Phi(z,y)}=0 
\end{equation}
を得る.よって$\qty{T_nf_k}_k$は$D$上の一様有界かつ同程度一様連続関数列であるから一様収束する部分列を持つ.したがって$T_n$はコンパクト作用素である.

(2) $f\in C(D),\ x\in D$に対して 
\begin{align}
    \abs{Tf(x)-T_nf(x)}
    &\leq \int_{D}\abs{x-y}^{-1}\qty(1-\Psi_{n}(x-y))\abs{f(y)}\,dy\\
    &\leq \norm{f}\int_{\qty{y\in D\mid \abs{x-y}\leq 2/n}}\abs{x-y}^{-1}\,dy\\
    &\leq \frac{4\pi}{n}\norm{f}
\end{align}
であるから,
\begin{equation}
    \norm{T-T_n}_{\mathcal{L}(C(D))}=\sup_{f\in C(D),\ \norm{f}\leq 1}\sup_{x\in D}\abs{Tf(x)-T_nf(x)}\leq \frac{4\pi}{n}\to0\quad (n\to 0)
\end{equation}
よってコンパクト作用素のノルム収束極限として$T$はコンパクト作用素である.
\subsubsection*{[8]}
(1) 部分積分により,$u\in C^2([-1,1])$と$h\in X_0$に対して
\begin{equation}
    \int_{-1}^{1} u(h''+h'+h)\,dx=\int_{-1}^{1} (u''-u'+u)h\,dx 
\end{equation}
である.よって次を示せばよい:$h\in C([-1,1])$が
\begin{equation}
    \int_{-1}^{1}fh\,dx=0 \quad (\forall h\in X_0)
\end{equation}
を満たすならば$f\equiv 0$である.
\begin{equation}
    \eta(x)=\begin{cases}
        \exp\qty(-\frac{1}{1-x^2}) & (\abs{x}<1) \\
        0 & (\abs{x}\geq 1)
    \end{cases},\quad \eta_n(x)=n\eta(nx)\quad (n\geq 1)
\end{equation}
とおくと$\eta_n\in C_c^\infty(B_{1/n}),\ \eta\geq0,\ \int \eta_n=1$である.$x_0\in(-1,1)$を固定する.$h(x)=\eta_n(x-x_0)$は,$h\in C_c^\infty(B_{1/n}(x_0))$より,$n$が十分大きければ$X_0$の元である.よって
\begin{align}
    \abs{f(x_0)}&=\abs{f(x_0)-\int_{B_{1/n}(x_0)}fh\,dx}\\
    &=\int_{B_{1/n}(x_0)}\abs{f(x_0)-f(x)}\eta_n(x-x_0)\,dx\\
    &\leq \sup_{x\in B_{1/n}(x_0)}\abs{f(x_0)-f(x)}\\
    &\to 0\quad (n\to\infty )
\end{align}
であるから$f(x_0)=0\quad (x_0\in(-1,1))$が示された.

(2) $u(x)=\frac{2a}{\sqrt{3}}e^{x/2}\sin\frac{\sqrt{3}}{2}x$.

(3) $u(x)=\begin{cases}
    \frac{2}{\sqrt{3}}e^{x/2}\sin \frac{\sqrt{3}}{2}x & (x\geq 0)\\
    0 & (x\leq 0)
\end{cases}$が答え.以下見つけ方:$u=0\ (x\leq 0)$の形で探す.部分積分より,$h\in X_0$に対して
\begin{equation}
    \int_{-1}^{1}u(h''+h'+h)\,dx=\int_{0}^{1}u(h''+h'+h)\,dx=-u(0)h(0)+u'(0)h(0)-u(0)h'(0)+\int_{0}^{1}(u''-u'+u)h\,dx 
\end{equation}
となる.よって問題の条件を満たす$u$は,
\begin{equation}
    u(0)=0,\ u'(0)=1,\quad u''-u'+u=0\ (x\geq0)
\end{equation}
を満たすものでよい.
\end{document}